%%
%% This is file `thesis.tex',
%% generated with the docstrip utility.
%%
%% The original source files were:
%%
%% nudtpaper.dtx  (with options: `thesis')
%% 
%% This is a generated file.
%% 
%% Copyright (C) 2012 by Liu Benyuan <liubenyuan@gmail.com>
%% 
%% This file may be distributed and/or modified under the
%% conditions of the LaTeX Project Public License, either version 1.3a
%% of this license or (at your option) any later version.
%% The latest version of this license is in:
%% 
%% http://www.latex-project.org/lppl.txt
%% 
%% and version 1.3a or later is part of all distributions of LaTeX
%% version 2004/10/01 or later.
%% 
%% To produce the documentation run the original source files ending with `.dtx'
%% through LaTeX.
%% 
%% Any Suggestions : LiuBenYuan <liubenyuan@gmail.com>
%% Thanks Xue Ruini <xueruini@gmail.com> for the thuthesis class!
%% Thanks sofoot for the original NUDT paper class!
%% 
%1. 规范硕士导言
% \documentclass[master,ttf]{nudtpaper}
%2. 规范博士导言
% \documentclass[doctor,twoside,ttf]{nudtpaper}
%3. 如果使用是Vista
% \documentclass[master,ttf,vista]{nudtpaper}
%4. 建议使用OTF字体获得较好的页面显示效果
%   OTF字体从网上获得,各个系统名称统一,不用加vista选项
%   如果你下载的是最新的(1201)OTF英文字体,建议修改nudtpaper.cls,使用
%   Times New Roman PS Std
% \documentclass[doctor,twoside,otf]{nudtpaper}
%5. 如果想生成盲评,传递anon即可,仍需修改个人成果部分
% \documentclass[master,otf,anon]{nudtpaper}
%
\documentclass[master,otf]{nudtpaper}
\usepackage{mynudt}

\classification{TP957}
\serialno{0123456}
\confidentiality{公开}
\UDC{}
%% \title{国防科大学位论文\LaTeX{}模板\\使用手册}
\title{函数式并行程序语言研究}
%% \displaytitle{国防科学技术大学学位论文\LaTeX{}模板}
\displaytitle{函数式并行程序语言研究}
\author{苏醒}
\zhdate{\zhtoday}
\entitle{Research on Parallel Functional Programming Language}
\enauthor{Su Xing}
\endate{\entoday}
\subject{计算机科学与技术}
\ensubject{Computer Science and Technology}
\researchfield{函数式并行程序语言}
\supervisor{窦文华\quad{}教授}
%% \cosupervisor{王五\quad{}副教授} % 没有就空着
\ensupervisor{Professor Dou Wenhua}
%% \encosupervisor{}
\papertype{工学}
\enpapertype{Engineering}
% 加入makenomenclature命令可用nomencl制作符号列表。

\begin{document}
\graphicspath{{figures/}}
% 制作封面,生成目录,插入摘要,插入符号列表 \\
% 默认符号列表使用denotation.tex,如果要使用nomencl \\
% 需要注释掉denotation,并取消下面两个命令的注释。 \\
% cleardoublepage% \\
% printnomenclature% \\
\maketitle
\frontmatter
\tableofcontents
\listoftables
\listoffigures

\midmatter
\begin{cabstract}
国防科学技术大学是一所直属中央军委的综合性大学。1984年,学校经国务院、中央军委和教育部批准首批成立研究生院,%
肩负着为全军培养高级科学和工程技术人才与指挥人才,培训高级领导干部,从事先进武器装备和国防关键技术研究的重要任务。%
国防科技大学是全国重点大学,也是全国首批进入国家“211工程” 建设并获中央专项经费支持的全国重点院校之一。%
学校前身是1953年创建于哈尔滨的中国人民解放军军事工程学院,简称“哈军工”。
\end{cabstract}
\ckeywords{国防科学技术大学; 211; 哈军工}

\begin{eabstract}
National University of Defense Technology is a comprehensive national key university based in Changsha, %
Hunan Province, China. It is under the dual supervision of the Ministry of National Defense %
and the Ministry of Education, designated for Project 211 and Project 985, %
the two national plans for facilitating the development of Chinese higher education. %

NUDT was originally founded in 1953 as the Military Academy of Engineering in Harbin of Heilongjiang Province. %
In 1970 the Academy of Engineering moved southwards to Changsha and was renamed Changsha Institute of Technology.%
 The Institute changed its name to National University of Defense Technology in 1978.

\end{eabstract}
\ekeywords{NUDT; MND; ME}


\begin{denotation}

\item[HPC] 高性能计算 (High Performance Computing)
\item[cluster] 集群
\item[Itanium] 安腾
\item[SMP] 对称多处理
\item[API] 应用程序编程接口
\item[PI]	聚酰亚胺
\item[MPI]	聚酰亚胺模型化合物,N-苯基邻苯酰亚胺
\item[PBI]	聚苯并咪唑
\item[MPBI]	聚苯并咪唑模型化合物,N-苯基苯并咪唑
\item[PY]	聚吡咙
\item[PMDA-BDA]	均苯四酸二酐与联苯四胺合成的聚吡咙薄膜
\item[$\Delta G$]  	活化自由能~(Activation Free Energy)
\item [$\chi$] 传输系数~(Transmission Coefficient)
\item[$E$] 能量
\item[$m$] 质量
\item[$c$] 光速
\item[$P$] 概率
\item[$T$] 时间
\item[$v$] 速度

\end{denotation}


%书写正文,可以根据需要增添章节; 正文还包括致谢,参考文献与成果
\mainmatter
\chapter{绪论}

\section{课题研究背景}

\subsection{计算机硬件系统的并行化与异构化趋势}
计算机行业从它诞生伊始就保持着日新月异的发展势头。摩尔定律(Moore's Law)FIXME:citehere的神奇预言
在过去的半个世纪中准确地反映了半导体与集成电路工业的发展速度:
集成电路上的晶体管数量每18个月翻一番。在摩尔定律的作用下,单个微处理器(microprocessor)的时钟频率
越来越高,处理器性能也随之不断提高。早期计算机均采用单一微处理器作为计算部件,其性能的提高主要依赖
微处理器运算速度的提高。但随着计算机行业持续快速的发展,仅仅提高单个处理器的运算速度已经不足以使
整个计算机系同的性能得到明显提升。

首先,单个处理器的性能提高有其极限,由于功耗问题FIXME:citehere以及晶体管器件本身物理特性的限制,
单个微处理器的频率不可能无限提高;其次,随着计算机系统结构日益复杂和多样化,
访存延迟、存储器带宽等因素可能成为制约性能的主要瓶颈;此外,程序中可发掘的指令级
并行(ILP)能力有限,单一指令流已经难以充分利用高速处理器的处理能力。
因此,当今业界主流厂商已经将眼光投向并行硬件,包括多处理器(multi-processor)计算机、
多核处理器(multi-core processor)以及面向特定应用的协处理器(coprocessor),
希望通过为用户提供多个处理器或为特定应用需求提供专用处理器来获得更高的性能。

2001年IBM推出了首款通用双核(dual-core)处理器,此后,各大主流厂商相继推了不同系列的多核处理器,单个
处理器包括的核心数目不等,有的处理器核心数目可以达到16或者更多(如AMD Opteron系列,Intel Xeon系列等)。
短短几年之内,多核处理器已经占据市场主流位置,从超级计算机到高性能服务器,从桌面PC到移动通信设备,
多核处理器在许多领域都得到了广泛应用。

协处理器一般是针对特定应用而设计的处理器,如浮点计算、图形处理、信号处理、加密解密等。相较于通用
处理器,协处理器可能缺乏某些功能,但在特定的功能上具备突出的性能优势。许多计算机系统都配备一个或
多个协处理器,用以辅助CPU执行计算。图形处理器(Graphics Processor Unit, GPU)是当前应用最为广泛的一类协处理器,
它最开始被专门设计于图形处理问题,协助CPU完成图形渲染等工作,但近几年GPU已经越来越成为一种“通用”计算设备,
主流GPU厂商已经为在GPU上执行通用计算提供支持软硬件支持采取了许多努力。
GPU采用流处理(stream processing)编程模型,这是一种类似于SIMD(Single Instruction Multiple Data)
的编程模型FIXME:citehere。在流处理模型中,一组数据中的每一个单元都被
施加同一操作,一般将这种操作称之为Kernel。Kernel可以是一段程序,允许有一定的程序逻辑,而不限于单一指令,
这比SIMD提供了更高的灵活性。在硬件设计上,GPU通常拥有数百甚至上千个核心,Kernel被流水化执行,这可以大大加速
数据级并行问题的执行速度。

随着多核处理器与面向特定问题的协处理器的兴起,由通用处理器与协处理器组成异构硬件系统
(heterogeneous architecture)也已经成为当前计算机系统的主流构成模式。
由国防科学技术大学研制,在2013年度Top500FIXME:citehere评比中
排名第一的天河二号超级计算机,就采用了这种异构系统设计:它的每个计算结点由两颗Xeon E5 12核处理器(CPU)
与三颗Xeon Phi 61核协处理器(GPU)组成,CPU+GPU的硬件配置为大规模科学计算提供了强大的驱动力。在桌面
计算机市场上,一般的PC机都配备了专门的图形处理器,这些现代图形处理器,除协助CPU完成
图形渲染方面的任务之外,也支持通用计算,允许PC机用户在显卡上进行通用计算编程。
最新的智能手机也已经开始采用GPU来辅助CPU执行计算,以期为游戏应用提供更强的计算支持。

在更加宏观的层次上,并行硬件一直是提供大规模计算能力的主要技术。互联网上存在着大量的服务器集群,由数十台甚至
上百台独立计算机通过通用网络互联,搭建成的集群系统为大规模事务处理提供了强力支持。由成百上千个高性能
计算结点通过专用网络互联组成的超级计算级,使大规模科学计算问题顺利执行。这些宏观计算机系统由独立的计算机结点
组成,每个结点独立并行地执行任务,每个结点又是由并行处理器、协处理器等并行硬件构成的异构系统。

总之,并行化与异构化是当今计算机硬件系统的主流发展趋势,在可以预见的将来,计算机性能的提高将以并行硬件
的驱动与异构硬件的利用为主要手段。

\subsection{并行化与异构化硬件的软件支持}
多核处理器与协处理器为设计实现更高性能的计算机系统提供了硬件支持,如何适应
硬件的发展,有效地利用这些硬件资源,为程序员提供易用的、高效的编程工具,是提高并行硬件应用能力
的关键问题。并行编程工具的设计需要兼顾两个核心特性:FIXME:emphasizehere高效性与易用性。
高效性是指并行编程工具必须能够充分开发并行硬件的计算能力,最大程度地利用硬件资源;
易用性是指在有效开发硬件并行能力的前提下,编程工具应该是易用的,并行程序设计的难度与复杂度不能过高。
当前已经有多种并行编程工具得到了广泛使用,这些工具有的
针对多核处理器设计,有的针对异构硬件设计,有的适用于数据并行问题,有的适用于任务并行问题。
这些广泛应用的并行编程技术包括多线程、消息传递接口、并行程序语言、编译器制导指令等,
它们特点各异,在通用性、适用问题方面都有不同,下面分别简要介绍。

多线程(multithreading)技术是在传统串行编程技术的基础上发展起来一种编程模型,
它的出现早于多核处理器。一个线程在逻辑上是一个独立指令流,不同线程的指令流可以在单一处理器上交替
执行以隐藏一些耗时较高的操作(如IO),避免处理器空转,提高处理器的吞吐率。
在多核处理器上,多个线程可以真正“并行”地执行。使用多线程编程模型,编程者显式地
将整个任务划分为独立的子任务,不同的子任务在不同的线程中执行,从而利用更多处理器资源,
缩短整个程序的运行时间。多线程支持通常由操作系统(Operating System)提供,是一种非常成熟的并行
编程技术,POSIXFIXME:citehere线程是多线程模型的POSIX标准,Pthreads在多种操作系统上均有实现。

消息传递接口(Message Passing Interface, MPIFIXME:citehere)是一套并行程序库接口规范,
它以多线程技术为基础,定义了线程间
通信的标准方法,最初定位用于分布式内存计算机系统,但也适用于其他体系结构的计算机系统。MPI定义的标准
通信接口功能全面,问题描述能力强大,具被性能高、可扩展性强、可移植性强的特点,长期以来一直都是高
性能计算领域的主要编程模型。

多线程技术与MPI技术都是在已有的串行程序语言之上,通过构建程序库的方式为并行编程提供支持。这种方式的
好处是硬件控制能力强,执行效率高,缺点是受限于已有的编程语言特性,细节隐藏能力差,编程复杂度高。
并行程序语言采用一种不同的思路,从语言层面为并行程序提供支持。并行程序语言一般通过提供特殊的并行语法
结构来表达程序中的并行部分,但在底层实现仍采用传统的多线程与MPI技术,任务并行化的工作由编译器自动完成。
并行编程语言的抽象层次一般较高,更加注重语言的易用性,对编程者隐藏并行硬件细节。并行程序语言一直
是并行软件技术的研究热点,这部分内容将在下一章着重介绍。

编译器制导指令(Compiler Directive)是一种介于并行程序库与并行程序语言之间的并行编程技术,
它允许程序员在源代码中插入专用的制导指令(directive)来指出
程序中的并行部分,并在必要之处加入同步互斥及通信。编译器识别这些制导指令,对程序做自动并行化处理。
编译器也可以视情况忽略这些制导指令,这时程序退化为串行程序。这种技术只需要程序员做简单的制导工作,
具有较好的可移植性,可以根据硬件资源的数量自动调整并行度,缺点在于难以调试,缺乏错误处理,对线程
粒度控制较弱,同时很难应用于非共享内存计算机系统。编译器制导指令技术的代表是OpenMPFIXME:citehere。

开发协处理器并行能力的技术在学术界与工业界都受到高度关注,发展也十分迅速。Nvidia公司针对自己的GPU
首先提出了CUDA编程架构,使用CUDA C语言对GPU进行编程,CUDA C是一种C语言的扩展语言FIXME:citehere。
CUDA C在一定程度上暴露了GPU的硬件细节,用户对程序行为的控制力强,程序性能较高。著名的
非盈利技术联盟Khronos Group针对异构计算提出了OpenCL标准FIXME:citehere,Nvidia与AMD公司分别在自己的GPU上支持OpenCL
实现。编译器制导指令技术在协处理器上也得到了应用,代表有OpenACC与OpenHMPP。FIXME:citehere

总体上,并行程序设计的难度远远高于串行程序设计。在基于传统串行程序语言的解决方案中,编程者不仅需要
关注于解决问题的程序逻辑,还需要显式地处理各种并行细节,诸如线程创建、内存管理、消息传递、任务划分等。
并行程序语言从根本上解决了编程难的问题,但现有的实现技术效果仍不理想,多数研究成果只能适用于
特定领域特定问题,通用性有限。相较于并行硬件的发展,并行软件技术的发展处于相对滞后的状态,现阶段
并行软件技术还没有一个完美的解决方案。高效性与易用性的平衡与兼顾仍然是一个亟待解决的关键问题。

\section{课题研究意义}
正如上一小节所说,计算机硬件系统结构的主流发展趋势是并行化与异构化,而开发并行化与异构化硬件的计算能力
需要相应的并行编程技术提供编程工具。未来计算机系统的性能提高,关键就在于如何从软硬件两方面开发计算机系统
的并行执行能力。这其中,并行编程工具既要适应硬件的特性以提高执行效率,又要独立于不同的硬件结构提供易用
的编程界面,高效性是基本要求,在保证高效性的前提下尽量兼顾易用性。

具体来说,软件工具需要解决的问题有下列三点,其中前两点是实现并行软件的高效性要求,第三点是易用性要求。
\begin{itemize}
  \item 如何最大限度地开发并行硬件(包括多核通用处理器与众核协处理器。)的计算能力?
  \item 如何设计编程模型,使异构系统中通用处理器与协处理器更加有效地协调配合?
  \item 如何对用户隐藏计算机系统的并行硬件细节,降低并行程序设计的难度?
\end{itemize}

本论文着眼于上述并行软件需要解决的三个问题,对并行程序语言展开研究,
设计了一门数据并行程序语言Rat,为编程者提供一个抽象层次高、表达能力强、细节隐藏好的编程界面,大大降低
并行程序设计的难度。并行程序语言从根本上解决并行编程难的问题,对降低并行计算机系统上的应用开发难度、
提高并行计算机可用性具有重要意义。

为了保证高效性,Rat语言的设计充分考虑了GPU的硬件特点,Rat语言编译得到的程序可以在GPU上高效执行。同时,
借鉴函数式程序语言的优良性质,提出一种驱动同异构硬件协同工作的有效方法,能够发掘独立于问题的
程序并行性。这对于利用未来计算机异构硬件系统也也具有重要意义。

采用Rat语言编写的并行程序运行在一个运行时系统之上,该运行时系统可以根据不同的硬件配置采用不同的
运行时策略,这样既保证了对硬件的充分利用,由保证了较强的可移植性,使Rat程序在不同硬件配置下均可
达到比较理想的性能。

\section{主要研究内容}
函数式并行程序语言Rat的研究内容主要分为两方面:前端语法设计与后端编译实现技术。其中前端语法设计着重实现
并行编程工具的易用性,后端编译实现技术着重实现并行编程工具的高效性。

\subsection{并行程序语言语法设计}
Rat语言的语法设计旨在为编程者提供一个通用、易用的并行编程模型,Rat语言具备以下优良特性:
\begin{itemize}
  \item 抽象层次高。编程者在编程解决问题时只需要关注问题本身的逻辑,无需关注底层的并行实现细节,线程管理、
    内存管理、任务划分等工作由编译器与运行时系统维护。
  \item 表达能力强。Rat精心选取了一组并行原语,使用这一组有限的并行原语就可以方便地描述一大类数据并行问题。
  \item 语法简洁精巧,易学易用。
\end{itemize}

\subsection{并行程序语言编译实现技术}
Rat语言的设计目标是兼顾易用性与高效性。易用性由它的语法设计提供,高效性则有赖于它的编译实现技术。
本论文在Rat语言的编译实现技术方面主要包括下列研究点:
\begin{itemize}
  \item 并行虚拟机的设计。并行虚拟机是对并行硬件的抽象,它运行精简的指令集,易于分析和优化,并且在
    并行硬件上有高效实现。
  \item 开发协处理器的并行计算能力。Rat的首个实现采用Nvidia公司的GPU作为并行计算硬件,作为并行虚拟机
    的实现平台。在编译实现过程中着重考虑如何充分利用GPU的流处理器资源、共享存储器资源,
    如何达到更高的访存带宽等,采取了一些优化技术。
  \item 开发异构硬件的系统工作能力。借鉴函数式程序语言的优良特性,对如何驱动CPU与GPU协同、并行工作展开研究,
    提出一种与具体问题无关的并行性发掘技术,使不同的处理器资源发挥各自所长,提高系统整体的性能。
\end{itemize}

\section{论文结构组织}
本论文设计并实现了一门并行程序语言Rat。全文组织结构如下:

第一章介绍了论文的研究背景,指出计算机硬件并行化与异构化的发展趋势,概括了并行软件技术的发展目标,
阐述了本课题的研究意义,简述了本课题的主要研究内容。

第二章介绍了并行软件技术领域的国内外研究现状,分析现有研究采用的方法,取得的成果以及存在的不足。

第三章详细说明了Rat语言的语法设计,分别就函数式语言特性、类型系统、向量原语展开描述。

第四章说明Rat语言编译实现所采用的关键技术,包括并行虚拟机的设计、并行虚拟机在并行硬件上
的实现与优化、异构硬件系统的协同工作技术研究与并行自动化技术研究等方面。

第五章FIXME:shiyan

第六章对论文的工作进行总结,指出了论文的不足以及未来的工作方向。

%% \begin{figure}[htp]
%% \centering
%% \includegraphics{picmain}
%% \caption{图 1.1 名称}
%% \end{figure}

%% \begin{table}[htp]
%% \centering
%% \caption{表 1.2 名称}
%% \begin{tabular}{|c|c|c|c|c|}
%% \hline
%% \makebox[2.07cm][0pt]{} & \makebox[2.07cm][0pt]{} & \makebox[2.07cm][0pt]{} & \makebox[2.07cm][0pt]{} & \makebox[2.07cm][0pt]{} \\
%% \hline
%%  & & & & \\
%% \hline
%%  & & & & \\
%% \hline
%% \end{tabular}
%% \end{table}


\chapter{论文正文}
\label{chap:main}
本章将进入论文排版的正文, 按元素分主要包括:
{\kai 字体段落,图片表格,公式定理,参考文献}这几部分。
这个样例文件将包括模板中使用到的所有格式、模板中自定义命令到或者特有的东西,
都将被一一介绍,希望大家在排版自己的学位论文前能细致的看一遍,记住样例的格式和
方法,方便上手。

\section{字体段落}
\label{sec:font}

陈赓(1903年2月27日-1961年3月16日),原名陈庶康,中国湖南湘乡人,军事家。出生将门,其祖父为湘军将领陈翼怀。

Adobe中文字体有四种:

{\kai 楷体\verb|\kai|:陈赓,中国湖南湘乡人,军事家。出生将门,其祖父为湘军将领陈翼怀。%
1952年筹办并任人民解放军军事工程学院第一任院长兼政委,培养国防科技人才。1955年被授予大将军衔。}

{\fs 仿宋\verb|\fs|:陈赓,中国湖南湘乡人,军事家。出生将门,其祖父为湘军将领陈翼怀。%
1952年筹办并任人民解放军军事工程学院第一任院长兼政委,培养国防科技人才。1955年被授予大将军衔。}

{\hei 黑体\verb|\hei|:陈赓,中国湖南湘乡人,军事家。出生将门,其祖父为湘军将领陈翼怀。%
1952年筹办并任人民解放军军事工程学院第一任院长兼政委,培养国防科技人才。1955年被授予大将军衔。}

宋体就是正文字体了。下面测试字体大小,\LaTeX{}默认的列表环境会在
条目之间插入过多的行距,在下面这种情况可能正好,若用户需要
{\kai 正文行距}的列表环境,可以使用compactitem环境,记住这点很重要,不要再
用那种自己修改\verb|itemsep|的傻傻的办法了。
\begin{itemize}
\item[初号] {\song\chuhao 陈赓大将}
\item[小初] {\song\xiaochu 陈赓大将}
\item[一号] {\song\yihao 陈赓大将}
\item[小一] {\song\xiaoyi 陈赓大将}
\item[二号] {\song\erhao 陈赓大将}
\item[小二] {\song\xiaoer 陈赓大将}
\item[三号] {\song\sanhao 陈赓大将}
\item[小三] {\song\xiaosan 陈赓大将}
\item[四号] {\song\sihao 陈赓大将}
\item[小四] {\song\xiaosi 陈赓大将}
\item[五号] {\song\wuhao 陈赓大将}
\item[小五] {\song\xiaowu 陈赓大将}
\end{itemize}

\section{表格明细}
\label{sec:figure}
表格是论文的重要组成部分,我们从简单的表格讲起,到复杂的表格为止。

模板中关于表格的宏包有三个: \textsf{booktabs}、\textsf{array} 和
\textsf{longtabular}。三线表建议使用\textsf{booktabs}中提供的,
包含toprule、midrule 和 bottomrule三条命令,简单干脆!
它们与\textsf{longtable} 能很好的配合使用。下面来看一个表格实例:
\begin{table}[htb]
  \centering
  \begin{minipage}[t]{0.8\linewidth} % 如果想在表格中使用脚注,minipage是个不错的办法
  \caption[模板文件]{模板文件。如果表格的标题很长,那么在表格索引中就会很不美
    观,所以要像 chapter 那样在前面用中括号写一个简短的标题。这个标题会出现在索
    引中。}
  \label{tab:template-files}
    \begin{tabular*}{\linewidth}{lp{10cm}}
      \toprule[1.5pt]
      {\hei 文件名} & {\hei 描述} \\
      \midrule[1pt]
      nudtpaper.ins & \LaTeX{} 安装文件,docstrip\footnote{表格中的脚注} \\
      nudtpaper.dtx & 所有的一切都在这里面\footnote{再来一个}。\\
      nudtpaper.cls & 模板类文件。\\
      nudtpaper.cfg & 模板配置文。cls 和 cfg 由前两个文件生成。\\
      bstutf8.bst   & 参考文献 Bibtex 样式文件。\\
      mynudt.sty    & 常用的包和命令写在这里,减轻主文件的负担。\\
      \bottomrule[1.5pt]
    \end{tabular*}
  \end{minipage}
\end{table}

表 \ref{tab:template-files} 列举了本模板主要文件及其功能,基本上来说论文
中最可能用到的就是这种表格形式了。
请大家注意三线表中各条线对应的命令。这个例子还展示了如何在表格中正确使用脚注。
如果你不需要在表格中插入脚注,可以将minipage环境去掉。
由于\LaTeX{}本身不支持在表格中使用\verb|\footnote|,所以我们不得不将表格放在
小页中,而且最好将表格的宽度设置为小页的宽度,这样脚注看起来才更美观。

另外六院的同学在使用模板时需要使用一种固定宽度(往往是页宽,下面的例子由
rongdonghu提供)的表格,内容需要居中且可以自动调整。
解决办法是自定义了一种\verb|tabularx|中的\textbf{Z}环境,在论文模板中,
该命令已添加到\verb|mynudt.sty|中。下面是这种情况的实例:

\begin{table}[htbp]
\centering
\begin{minipage}[t]{0.9\linewidth}
\caption{Reed Solomon码的典型应用}
\label{tab:RSuse}
\begin{tabularx}{\linewidth}{cZ}
\toprule[1.5pt]
{\hei 应用领域} & {\hei 编码方案}\\
\midrule[1pt]
磁盘驱动器 & RS(32,28,5)码 \footnote{码长为32、维数为28、最小距离为5} \\
CD & 交叉交织RS码(CIRC) \\
DVD & RS(208,192,17)码、RS(182,172,11)码 \\
光纤通信 & RS(255,229,17)码 \\
\bottomrule[1.5pt]
\end{tabularx}
\end{minipage}
\end{table}

我们经常会在表格下方标注数据来源,或者对表格里面的条目进行解释。前面的脚注是一种
不错的方法,如果你不喜欢minipage方法的脚注。
那么完全可以在表格后面自己写注释,比如表~\ref{tab:tabexamp1}。
\begin{table}[htbp]
  \centering
  \caption{复杂表格示例 1}
  \label{tab:tabexamp1}
  \begin{minipage}[t]{0.8\textwidth} 
    \begin{tabularx}{\linewidth}{|l|X|X|X|X|}
      \hline
      \multirow{2}*{\backslashbox{x}{y}}  & \multicolumn{2}{c|}{First Half} & \multicolumn{2}{c|}{Second Half}\\
      \cline{2-5}
      & 1st Qtr &2nd Qtr&3rd Qtr&4th Qtr \\ 
      \hline
      \multirow{2}*{East$^{*}$} &   20.4&   27.4&   90&     20.4 \\
       &   30.6 &   38.6 &   34.6 &  31.6 \\ 
      West$^{**}$ &   30.6 &   38.6 &   34.6 &  31.6 \\ 
      \hline
    \end{tabularx}\\[2pt]
    \footnotesize
    *:东部\\
    **:西部
  \end{minipage}
\end{table}

此外,表~\ref{tab:tabexamp1} 同时还演示了另外三个功能:1)通过 \textsf{tabularx} 的
 \texttt{|X|} 扩展实现表格内容自动调整;2)通过命令 \verb|\backslashbox| 在表头部分
插入反斜线(WORD中很简单,但\LaTeX{}做表格需要一定的(极大的)想象力);3)就是
使用\verb|multirow|和\verb|multicolumn|命令。

不可否认 \LaTeX{} 的表格功能没有想象中的那么强大,不过只要你足够认真,足够细致,那么
同样可以排出来非常复杂非常漂亮的表格。可是科技论文中那么复杂表格有什么用呢?
上面那个表格就够用啦。

浮动体的并排放置一般有两种情况:1)二者没有关系,为两个独立的浮动体;2)二者隶属
于同一个浮动体。对表格来说并排表格既可以像表~\ref{tab:parallel1}、表~\ref{tab:parallel2} 
使用小页环境,也可以如表~\ref{tab:subtable}使用子表格来做。
图与表同出一源,后面我们将讲解子图(subfloat)的例子。
\begin{table}[htb]
\centering
\noindent\begin{minipage}{0.45\textwidth}
\centering
\caption{第一个并排子表格}
\label{tab:parallel1}
\begin{tabular}{p{2cm}p{2cm}}
\toprule[1.5pt]
111 & 222 \\\midrule[1pt]
222 & 333 \\\bottomrule[1.5pt]
\end{tabular}
\end{minipage}
\begin{minipage}{0.45\textwidth}
\centering
\caption{第二个并排子表格}
\label{tab:parallel2}
\begin{tabular}{p{2cm}p{2cm}}
\toprule[1.5pt]
111 & 222 \\\midrule[1pt]
222 & 333 \\\bottomrule[1.5pt]
\end{tabular}
\end{minipage}
\end{table}
\begin{table}[htbp]
\centering
\caption{并排子表格}
\label{tab:subtable}
\subfloat[第一个子表格]{
\begin{tabular}{p{2cm}p{2cm}}
\toprule[1.5pt]
111 & 222 \\\midrule[1pt]
222 & 333 \\\bottomrule[1.5pt]
\end{tabular}}\hskip2cm
\subfloat[第二个子表格]{
\begin{tabular}{p{2cm}p{2cm}}
\toprule[1.5pt]
111 & 222 \\\midrule[1pt]
222 & 333 \\\bottomrule[1.5pt]
\end{tabular}}
\end{table}

如果您要排版的表格长度超过一页,那么推荐使用\textsf{longtable}命令。
这里随便敲入一些无关的文字,使得正文看上去不是那么的少。
表~\ref{tab:performance} 就是 \textsf{longtable} 的简单示例。
\begin{longtable}[c]{c*{6}{r}}
\caption{实验数据}\label{tab:performance}\\
\toprule[1.5pt]
 测试程序 & \multicolumn{1}{c}{正常运行} & \multicolumn{1}{c}{同步}
& \multicolumn{1}{c}{检查点}   & \multicolumn{1}{c}{卷回恢复}
& \multicolumn{1}{c}{进程迁移} & \multicolumn{1}{c}{检查点} 	\\
& \multicolumn{1}{c}{时间 (s)} & \multicolumn{1}{c}{时间 (s)}
& \multicolumn{1}{c}{时间 (s)} & \multicolumn{1}{c}{时间 (s)}
& \multicolumn{1}{c}{时间 (s)} &  文件(KB)			\\
\midrule[1pt]%
\endfirsthead%

\multicolumn{7}{c}{续表~\thetable\hskip1em 实验数据}\\

\toprule[1.5pt]
 测试程序 & \multicolumn{1}{c}{正常运行} & \multicolumn{1}{c}{同步} 
& \multicolumn{1}{c}{检查点}   & \multicolumn{1}{c}{卷回恢复}
& \multicolumn{1}{c}{进程迁移} & \multicolumn{1}{c}{检查点} 	\\
& \multicolumn{1}{c}{时间 (s)} & \multicolumn{1}{c}{时间 (s)}
& \multicolumn{1}{c}{时间 (s)} & \multicolumn{1}{c}{时间 (s)}
& \multicolumn{1}{c}{时间 (s)} &  文件(KB)			\\
\midrule[1pt]%
\endhead%
\hline%

\multicolumn{7}{r}{续下页}%

\endfoot%
\endlastfoot%
CG.A.2 & 23.05   & 0.002 & 0.116 & 0.035 & 0.589 & 32491  \\
CG.A.4 & 15.06   & 0.003 & 0.067 & 0.021 & 0.351 & 18211  \\
CG.A.8 & 13.38   & 0.004 & 0.072 & 0.023 & 0.210 & 9890   \\
CG.B.2 & 867.45  & 0.002 & 0.864 & 0.232 & 3.256 & 228562 \\
CG.B.4 & 501.61  & 0.003 & 0.438 & 0.136 & 2.075 & 123862 \\
CG.B.8 & 384.65  & 0.004 & 0.457 & 0.108 & 1.235 & 63777  \\
MG.A.2 & 112.27  & 0.002 & 0.846 & 0.237 & 3.930 & 236473 \\
MG.A.4 & 59.84   & 0.003 & 0.442 & 0.128 & 2.070 & 123875 \\
MG.A.8 & 31.38   & 0.003 & 0.476 & 0.114 & 1.041 & 60627  \\
MG.B.2 & 526.28  & 0.002 & 0.821 & 0.238 & 4.176 & 236635 \\
MG.B.4 & 280.11  & 0.003 & 0.432 & 0.130 & 1.706 & 123793 \\
MG.B.8 & 148.29  & 0.003 & 0.442 & 0.116 & 0.893 & 60600  \\
LU.A.2 & 2116.54 & 0.002 & 0.110 & 0.030 & 0.532 & 28754  \\
LU.A.4 & 1102.50 & 0.002 & 0.069 & 0.017 & 0.255 & 14915  \\
LU.A.8 & 574.47  & 0.003 & 0.067 & 0.016 & 0.192 & 8655   \\
LU.B.2 & 9712.87 & 0.002 & 0.357 & 0.104 & 1.734 & 101975 \\
LU.B.4 & 4757.80 & 0.003 & 0.190 & 0.056 & 0.808 & 53522  \\
LU.B.8 & 2444.05 & 0.004 & 0.222 & 0.057 & 0.548 & 30134  \\
EP.A.2 & 123.81  & 0.002 & 0.010 & 0.003 & 0.074 & 1834   \\
EP.A.4 & 61.92   & 0.003 & 0.011 & 0.004 & 0.073 & 1743   \\
EP.A.8 & 31.06   & 0.004 & 0.017 & 0.005 & 0.073 & 1661   \\
EP.B.2 & 495.49  & 0.001 & 0.009 & 0.003 & 0.196 & 2011   \\
EP.B.4 & 247.69  & 0.002 & 0.012 & 0.004 & 0.122 & 1663   \\
EP.B.8 & 126.74  & 0.003 & 0.017 & 0.005 & 0.083 & 1656   \\
\bottomrule[1.5pt]
\end{longtable}

另外,有的同学不想让某个表格或者图片出现在索引里面,那么请使用命令 \verb|\caption*{}|,
这个命令不会给表格编号,也就是出来的只有标题文字而没有“表~XX”,“图~XX”,否则
索引里面序号{\kai 不连续}就显得不伦不类,这也是 \LaTeX{} 里星号命令默认的规则。

\section{绘图插图}

本模板不再预先装载任何绘图包(如 \textsf{pstricks,pgf} 等),完全由你自己来决定。
个人觉得 \textsf{pgf} 不错,不依赖于 Postscript。此外还有很多针对 \LaTeX{} 的
 GUI 作图工具,如 XFig(jFig), WinFig, Tpx, Ipe, Dia, Inkscape, LaTeXPiX,
jPicEdt 等等。本人强烈推荐\textsf{Ipe}。

一般图形都是处在浮动环境中。之所以称为浮动是指最终排版效果图形的位置不一定与源文
件中的位置对应,这也是刚使用 \LaTeX{} 同学可能遇到的问题。
如果要强制固定浮动图形的位置,请使用 \textsf{float} 宏包,
它提供了 \texttt{[H]}(意思是图片就给我放在这里\textcolor{red}{H}ere)参数,
但是除非特别需要,不建议使用\texttt{[H]},而是推荐使用\texttt{[htbp]},
给\LaTeX{}更多选择。比如图~\ref{fig:ipe}。
\begin{figure}[htbp] % use float package if you want it here
  \centering
  \includegraphics[width=3in]{hello}
  \caption{利用IPE制图}
  \label{fig:ipe}
\end{figure}

若子图共用一个计数器,
那么请看图~\ref{fig:big1},它包含两个小图,分别是图~\ref{fig:subfig1} 
和图~\ref{fig:subfig2}。这里推荐使用\verb|\subfloat|,{\bf 不要再用}
\verb|\subfigure|和\verb|\subtable|。
\begin{figure}[htb]
  \centering%
  \subfloat[第一个小图形]{%
    \label{fig:subfig1}
    \includegraphics[height=2cm]{xh}}\hspace{4em}%
  \subfloat[第二个小图形。如果标题很长的话,它会自动换行,这个 caption 就是这样的例子]{%
    \label{fig:subfig2}
    \includegraphics[height=2cm]{xhh}}
  \caption{包含子图形的大图形}
  \label{fig:big1}
\end{figure}

而下面这个例子显示并排$3\times2$的图片,见图\ref{fig:subfig:3x2}:
\begin{figure}[htb]
\centering
\subfloat[]{\includegraphics[width=.27\textwidth]{typography}} \qquad
\subfloat[]{\includegraphics[width=.27\textwidth]{typography}} \qquad
\subfloat[]{\includegraphics[width=.27\textwidth]{typography}} \qquad
\subfloat[]{\includegraphics[width=.27\textwidth]{typography}} \qquad
\subfloat[]{\includegraphics[width=.27\textwidth]{typography}} \qquad
\subfloat[]{\includegraphics[width=.27\textwidth]{typography}}
\caption{并排图片}
\label{fig:subfig:3x2}
\end{figure}

要注意,图\ref{fig:subfig:3x2}例中
\texttt{qquad}相当于\verb|\hspace{2em}|,也就是2个字符的宽度,约0.08倍页宽,
图片宽度设定为0.27倍页宽是合适的;在该环境中,尽量不要手动换行,所以,不妨自己计算一下!

如果要把编号的两个图形并排,那么小页(minipage)就非常有用了,可以分别参考
图\ref{fig:parallel1}和图\ref{fig:parallel2}。其实这个例子和表格一节中并排
放置的表格一摸一样。
\begin{figure}[htb]
\begin{minipage}{0.48\textwidth}
  \centering
  \includegraphics[height=1.2cm]{xhh}
  \caption{并排第一个图}
  \label{fig:parallel1}
\end{minipage}\hfill
\begin{minipage}{0.48\textwidth}
  \centering
  \includegraphics[height=1.2cm]{xhh}
  \caption{并排第二个图}
  \label{fig:parallel2}
\end{minipage}
\end{figure}

图形就说这么多,因为大家在写论文是遇到的最大问题不是怎么把图插进去,
而是怎样做出专业的、诡异的、震撼的图片来,记得在这时参考前面推荐的那
些工具吧,当然必不可少的是Matlab了,至于如何加入中文标注、支持中文等等
可以上网去查,但这里{\kai 推荐一点},用好export命令,使得插入图片时尽可能的不要
缩放,保证图文的一致性。

\section{公式定理}
\label{sec:equation}
贝叶斯公式如式~(\ref{equ:chap1:bayes}),其中$p(y|\mathbf{x})$为后验;
$p(\mathbf{x})$为先验;分母$p(\mathbf{x})$ 为归一化因子,这是
实际应用中十分恐怖的一个积分式。
\begin{equation}
\label{equ:chap1:bayes}
p(y|\mathbf{x}) = \frac{p(\mathbf{x},y)}{p(\mathbf{x})}=
\frac{p(\mathbf{x}|y)p(y)}{p(\mathbf{x})} 
\end{equation}

论文里面公式越多,\TeX{} 就越 happy。再看一个 \textsf{amsmath} 的例子:
\newcommand{\envert}[1]{\left\lvert#1\right\rvert} 
\begin{equation}\label{detK2}
\det\mathbf{K}(t=1,t_1,\dots,t_n)=\sum_{I\in\mathbf{n}}(-1)^{\envert{I}}
\prod_{i\in I}t_i\prod_{j\in I}(D_j+\lambda_jt_j)\det\mathbf{A}
^{(\lambda)}(\overline{I}|\overline{I})=0.
\end{equation} 

大家在写公式的时候一定要好好看\textsf{amsmath}的文档,并参考模板中的用法:
\begin{multline*}%\tag{[b]} % 这个出现在索引中的
\int_a^b\biggl\{\int_a^b[f(x)^2g(y)^2+f(y)^2g(x)^2]
 -2f(x)g(x)f(y)g(y)\,dx\biggr\}\,dy \\
 =\int_a^b\biggl\{g(y)^2\int_a^bf^2+f(y)^2
  \int_a^b g^2-2f(y)g(y)\int_a^b fg\biggr\}\,dy
\end{multline*}

再看\ref{equ:split}:
\begin{equation}\label{equ:split}
\begin{split}
C(z) &= [z^n] \biggl[\frac{e^{3/4}}{\sqrt{1-z}} +
e^{-3/4}(1-z)^{1/2} + \frac{e^{-3/4}}{4}(1-z)^{3/2}
+ O\Bigl( (1-z)^{5/2}\Bigr)\biggr] \\
&= \frac{e^{-3/4}}{\sqrt{\pi n}} - \frac{5e^{-3/4}}{8\sqrt{\pi
n^3}} + \frac{e^{-3/4}}{128 \sqrt{\pi n^5}} +
O\biggl(\frac{1}{\sqrt{\pi
n^7}}\biggr)
\end{split}
\end{equation}

当然了,数学中必不可少的是定理和证明:
\begin{theorem}
  \label{chapTSthm:rayleigh solution}
  假定 $X$ 的二阶矩存在:
  \begin{equation}
         O_R(\mathbf{x},F)=\sqrt{\frac{\mathbf{u}_1^T\mathbf{A}\mathbf{u}_1} {\mathbf{u}_1^T\mathbf{B}\mathbf{u}_1}}=\sqrt{\lambda_1},
  \end{equation}
  其中 $\mathbf{A}$ 等于 $(\mathbf{x}-EX)(\mathbf{x}-EX)^T$,$\mathbf{B}$ 表示协方差阵 $E(X-EX)(X-EX)^T$,$\lambda_1$
$\mathbf{u}_1$是$\lambda_1$对应的特征向量,
\end{theorem}

对于希腊符号使用\verb|mathbf|命令可能有些问题,所以建议对符号
用\verb|bm|加粗,记得用\verb|\up<greek>|切换正体符号,下面看几个例子:
\verb|\gamma|斜体代表变量$\gamma$,\verb|\bm{\upgamma}|正体代表向量$\bm{\upgamma}$,
。\verb|\Gamma|正体代表操作符号$\Gamma$,
\verb|\bm{\Gamma}|正体粗体代表矩阵形式$\bm{\Gamma}$,
\verb|\varGamma|斜体代表变量$\varGamma$。另外对于大小写斜体的加粗可以见$\bm{\gamma}$和$\bm{\varGamma}$,
但是这两种科技论文中很少出现,这里只做测试。
非符号普通向量就用\verb|\mathbf|吧:$\mathbf{x}_k,\mathbf{X}_k$。
完整测试如下$\omega,\bm{\omega},\upomega,\bm{\upomega},\Omega,\bm{\Omega},\varOmega,\bm{\varOmega}$。

\begin{proof}
上述优化问题显然是一个Rayleigh商问题。我们有
  \begin{align}
     O_R(\mathbf{x},F)=\sqrt{\frac{\mathbf{u}_1^T\mathbf{A}\mathbf{u}_1} {\mathbf{u}_1^T\mathbf{B}\mathbf{u}_1}}=\sqrt{\lambda_1},
 \end{align}
 其中 $\lambda_1$ 下列广义特征值问题的最大特征值:
$$
\mathbf{A}\mathbf{z}=\lambda\mathbf{B}\mathbf{z}, \mathbf{z}\neq 0.
$$
 $\mathbf{u}_1$ 是 $\lambda_1$对应的特征向量。结论成立。
\end{proof}

下面来看看算法环境的定义和使用。
我们知道,故障诊断的最终目的,是将故障定位到部件,而由于信号--部件依赖矩阵的存在,因此,实质性的工作是找出由故障部件发出异常信号,
不妨称为源异常信号,而如前所述,源异常信号与异常信号依赖矩阵$\mathbf{S_a}$的全零列是存在一一对应的关系的。因此,我们只要获得了$\mathbf{S_a}$的全零列的相关信息,
也就获得了源异常信号的信息,从而能进一步找到故障源。
通过以上分析,我们构造算法\ref{alg53},用于实现非回路故障诊断。
%% \begin{algorithm}[htbp]
%%   \caption{非回路故障诊断算法}
%%   \label{alg53}
%%   \begin{algorithmic}[1]
%%     \REQUIRE 信号--部件依赖矩阵$\mathbf{A}$,信号依赖矩阵$\mathbf{S}$,信号状态向量$\alpha$
%%     \ENSURE 部件状态向量$\gamma$
%%     \STATE $\mathbf{P}\leftarrow\left(<\alpha>\right)$
%%     \STATE $\mathbf{S_{a}}\leftarrow\mathbf{P^T}\mathbf{S}\mathbf{P}$
%%     \FOR{$i=1$ to $S_a$的阶数$m$}
%%     \STATE $s_i\leftarrow s_i$的第$i$个行向量
%%     \ENDFOR
%%     \STATE $\beta_a\leftarrow\lnot \left(s_1\lor s_2\lor \cdots\lor s_m\right)^T$
%%     \STATE $\beta\leftarrow\mathbf{P}\beta_a$
%%     \STATE $\gamma\leftarrow\mathbf{A}\beta$
%%   \end{algorithmic}
%% \end{algorithm}

第一类故障回路推理与非回路故障推理是算法基本相同,稍微不同的是$\beta_a$的计算。因为第一类故障回路中的信号全部可能是源异常信号,因此我们不必计算
$\beta_a=\lnot \left(\left[s_1\lor s_2\lor \cdots\lor s_m\right]^T\right)$,而直接取$\beta_a=\underbrace{\left[\begin{array}{cccc}1&1&\cdots&1\end{array}\right]^T}_m$,将$\beta_a$代入
算法\ref{alg53},有
\[\beta=\mathbf{P}\beta_a=\mathbf{P}\underbrace{\left[\begin{array}{cccc}1&1&\cdots&1\end{array}\right]^T}_m=\alpha\]
因此一类故障回路的推理算法变得相当简单,例如算法\ref{alg54}
%% \begin{algorithm}[htbp]
%%   \caption{第一类故障回路诊断算法}
%%   \label{alg54}
%%   \begin{algorithmic}[1]
%%     \REQUIRE 信号--部件依赖矩阵$\mathbf{A}$,信号状态向量$\alpha$
%%     \ENSURE 部件状态向量$\gamma$
%%     \STATE $\gamma\leftarrow\mathbf{A}\alpha$
%%   \end{algorithmic}
%% \end{algorithm}

\section{参考文献}
\label{sec:bib}
当然参考文献可以直接写 bibitem,虽然费点功夫,但是好控制,各种格式可以自己随意改
写,在nudtpaper里面,建议使用JabRef编辑和管理文献,再结合\verb|bstutf8.bst|,
对中文的支持非常不错,格式也很规范。

本模板推荐使用 BIB\TeX,样式文件为 bstutf8.bst,符合学校的参考文献格式(如专利
等引用未加详细测试)。看看这个例子,关于书的\upcite{tex, companion},
还有这些\upcite{Krasnogor2004e, clzs, zjsw},关于杂志的\upcite{ELIDRISSI94,
  MELLINGER96, SHELL02},硕士论文\upcite{zhubajie, metamori2004},博士论文
\upcite{shaheshang, FistSystem01},标准文件\upcite{IEEE-1363},会议论文\upcite{DPMG,kocher99},%
技术报告\upcite{NPB2}。中文参考文献\upcite{cnarticle}\textsf{特别注意},需要在\verb|bibitem|中
增加\verb|language|域并设为\verb|zh|,英文此项可不填,之后由\verb|bstutf8|统一处理
(具体就是决定一些文献在中英文不同环境下的显示格式,如等、etc)。
若使用\verb|JabRef|,则你可按下面步骤来设置:
选择\textsf{Options}$\rightarrow$\textsf{Set Up General Fields},
在\verb|General:|后加入\verb|language|就可以了。

有时候不想要上标,那么可以这样 \cite{shaheshang},这个非常重要。

\section{代码高亮}
有些时候我们需要在论文中引入一段代码,用来衬托正文的内容,或者体现关键思路的实现。
在模板中,统一使用\texttt{listings}宏包,并且设置了基本的内容格式,并建议用户只
使用三个接口,分别控制:编程语言,行号以及边框。简洁达意即可,下面分别举例说明。

首先是设定语言,来一个C的,使用的是默认设置:
\begin{lstlisting}[language=C]
void sort(int arr[], int beg, int end)
{
  if (end > beg + 1)
  {
    int piv = arr[beg], l = beg + 1, r = end;
    while (l < r)
    {
      if (arr[l] <= piv)
        l++;
      else
        swap(&arr[l], &arr[--r]);
    }
    swap(&arr[--l], &arr[beg]);
    sort(arr, beg, l);
    sort(arr, r, end);
  }
}
\end{lstlisting}

当我们需要高亮Java代码,不需要行号,不需要边框时,可以:
\begin{lstlisting}[language=Java,numbers=none,frame=none]
// A program to display the message
// "Hello World!" on standard output

public class HelloWorld {
 
   public static void main(String[] args) {
      System.out.println("Hello World!");
   }
      
}   // end of class HelloWorld
\end{lstlisting}

细心的用户可能发现,行号被放在了正文框之外,事实上这样是比较美观的,
如果有些用户希望在正文框架之内布置所有内容,可以:
\begin{lstlisting}[language=perl,xleftmargin=2em,framexleftmargin=1.5em]
#!/usr/bin/perl
print "Hello, world!\n";
\end{lstlisting}

好了,就这么多,\texttt{listings}宏包的功能很强大也很复杂,如果需要自己定制,
可以查看其手册,耐心阅读总会找到答案。
\textbf{注意:} 当前代码环境中文注释的处理还不是很完善,对于注释请妥善处理。
在本模板中,推荐算法环境或者去掉中文的listings代码环境。
如果需要包含中文注释,不要求代码高亮,
就用\texttt{code}环境,这个环境是Verbatim的定制版,简单有效,
调用的是fancyvbr宏包,用户可在mynudt.sty中修改它的外观等等。
这里我们还可以给代码加上标签。
\begin{code}[label=hello.c]
public class HelloWorld {
   public static void main(String[] args) {
      System.out.println("Hello World!");
   }
}   // 世界,你好!
\end{code}

\section{符号列表}

{\hei 前面的话:}{\kai\color{blue} 
2.2版本后默认使用nomencl环境,如果你还是希望使用传统的\verb|definition.tex|,那么只需注释掉
顶层文件中的nomenclature即可。}

符号列表使用的是\verb|nomencl|包,自己简单定制了下,使用方法分为四步:
\begin{compactenum}
\item 将\verb|\makenomenclature|语句放在正文前,即\verb|\begin{document}|前面;
\item 将\verb|\printnomenclature|放在论文中,我在例子中将符号列表放在了英文摘要的
后面,正文第一章的前面,当然,你可以根据自己的需要或者教研室的规范放置在合理的位置上,
为了页面引用的正确,在这句话前面放上\verb|\cleardoublepage|;
\item 使用\verb|\nomenclature|命令在论文的各个位置上添加符号定义,语法后面会讲到;
\item 编译。编译需要首先运行一遍xelatex,之后运行
\begin{code}
makeindex -s nomencl.ist -o thesis.nls thesis.nlo
\end{code}
\end{compactenum}

你可以把这句编译命令放在\verb|makepdf.bat|中第一个\verb|xelatex thesis|下面。然后
双击\verb|makepdf.bat|就可以了,论文模板中已经为你添加上了,如果你强烈不想使用
nomencl环境,只要把它注释掉(前面加\verb|rem|)就可以。
另外,由于我使用的是VIM来编辑\TeX{}代码,具体到每个编辑器(诸如WinEDT,TeXWorks等)
如何设定该命令的快捷按钮,诸位可以搜索网上的教程。

下面简单说明下\verb|\nomenclature|命令,语法为。这里插入一些随机的文字,希望
对你在阅读帮助中的思维没有什么不良的影响。
\begin{code}
\nomenclature[<prefix>]{<symbol>}{<desc>}{<null>}
\end{code}
\verb|nomencl|模板的默认排序方法可能(大多都)不满足要求,
论文模板里,我们通过设定\verb|<prefix>|来实现符号列表的排序。
它分为两部分,比如如\verb|[Aa]|,第一个字母的含义是:
\begin{compactitem}
\item[`A'] 符号归为拉丁字母
\item[`G'] 希腊字母
\item[`X'] 上标
\item[`Z'] 下标
\end{compactitem}
每个标识后边的字幕\verb|a-z|作为当前符号组内的排列顺序,比如$\beta$就可以写成
\verb|[Gb]|,诸如此类。当然你一定注意到了,这个排序分组的设定只是为了记忆
方便,并不是强制的,因此你可以有自己的方案,比如Z是Greek,
R是Roman什么的,只要统一就好,只需记住,组间排列是按字母顺序排的。

注意符号表分四列,前三列的含义与命令中相同,
最后一列是符号定义时所在的页码。效果看例子,对于下式:
\begin{equation}\label{eq:heatflux}
   \dot{Q} = k \cdot A \cdot \Delta T
\end{equation}%
\nomenclature[Aq]{$\dot{Q}$}{heat flux}{}%
\nomenclature[Ak]{$k$}{overall heat transfer coefficient,式\eqref{eq:ohtc}}{}%
\nomenclature[Aa]{$A$}{area}{}%
\nomenclature[Al]{$L$}{length}{}%
\nomenclature[At]{$T$}{temperature}{}%
\nomenclature[At]{$\Delta T$}{temperature difference}{}%
\nomenclature[Gr]{$\gamma$}{中文测试, 以及一句很长的物理意义,很有可能超过当前栏的宽度,主要目的是看一看会不会出现某些异常情况。}{}%

或者:
\begin{equation}\label{eq:ohtc}
    \frac{1}{k} = \left[\frac{1}{\alpha _{\mathrm{i}}\,r_{\mathrm{i}}} +
    \sum^n_{j=1}\frac{1}{\lambda _j}\,
    \ln \frac{r_{\mathrm{a},j}}{r_{\mathrm{i},j}} +
    \frac{1}{\alpha _{\mathrm{a}}\,
    r_{\mathrm{a}}}\right] \cdot r_{\mathrm{reference}}
\end{equation}%
\nomenclature[Ga]{$\alpha$}{convection heat transfer coefficient}{}%
\nomenclature[Zi]{i}{in}{}%
\nomenclature[Gl]{$\lambda$}{thermal conductivity}{}%
\nomenclature[Za]{a}{out}{}%
\nomenclature[Zn]{$n$}{number of walls}{}%
\nomenclature[Zj]{$j$}{running parameter}{}%

{\hei 注意事项:}{\kai 模板中定制的nomencl格式在mynudt.sty中,默认是三栏的,分别是:
``符号'',``定义'',``首次出现页码'',
注意这里的符号列表都没有单位,如果你需要额外的栏输入单位(呵呵,聪明的读者可能看出来
了,\verb|nomenclature|命令最后一个是空的,就是用来让你赋予她各种意义的)。
此时就需要你有一点点动手能力了(其实只要会修改表格就行),
方法很简单,比如需要添加``国际单位制''这一栏,则
\begin{compactenum}
\item 论文中\verb|\nomenclature|命令的第三个参数就让他代表单位,也可留空;
\item 将\verb|mynudt.sty|中longtable的表头添加``国际单位制''几个字,
你也可以取其他的名字,放在那个{\kai 应该出现的}位置上;
\item 由于增加了5个字,就把前面栏的宽度数字减5,同时设定第三栏宽度为5,
注意这一步需要你自己调整,记得不要让表格超出边界就行。
\end{compactenum}
}

\section{中文习惯}
\label{sec:chinese}

对于itermize过大的行间距,用户可以使用compactitem环境来替代,但是模板中不进行默认替代,
因为只有用户真正发现列表不好看才会找到这里,而且在示例文件中,
陈赓大将那个列表环境如果压缩了行距会很不好看。谢谢ZhangLei的建议!

{\hei 一个重要的提示:}
作者自己的定义命令、包等,不要放在模板里面,请放到\verb|mynudt.sty|
中,这样模板时,只要覆盖\verb|nudtpaper.cls|即可。

中文破折号为一个两个字宽垂直居中的直线,输入法直接得到的破折号是两个断开的小短线
(——),这看起来不舒服。所以模板中定义了一个破折号的命令 \verb|\pozhehao|,请看:

厚德博学,强军兴国\hfill \pozhehao{}国防科大校训




%%% Local Variables:
%%% mode: latex
%%% TeX-master: "../main"
%%% End:

\begin{ack}
  衷心感谢导师 xxx 教授和 xxx 副教授对本人的精心指导。他们的言传身教将使我终生受益。

  感谢 \nudtpaper{},它的存在让我的论文写作轻松自在了许多,让我的论文格式规整漂亮了许多。

\end{ack}


\cleardoublepage
\phantomsection
\addcontentsline{toc}{chapter}{参考文献}
\bibliographystyle{bstutf8}
\bibliography{ref/refs}

\begin{resume}

  \section*{发表的学术论文} % 发表的和录用的合在一起

  \begin{enumerate}[{[}1{]}]
  \addtolength{\itemsep}{-.36\baselineskip}%缩小条目之间的间距,下面类似
  \item Yang Y, Ren T L, Zhang L T, et al. Miniature microphone with silicon-
    based ferroelectric thin films. Integrated Ferroelectrics, 2003,
    52:229-235. (SCI 收录, 检索号:758FZ.)
  \item 杨轶, 张宁欣, 任天令, 等. 硅基铁电微声学器件中薄膜残余应力的研究. 中国机
    械工程, 2005, 16(14):1289-1291. (EI 收录, 检索号:0534931 2907.)
  \item 杨轶, 张宁欣, 任天令, 等. 集成铁电器件中的关键工艺研究. 仪器仪表学报,
    2003, 24(S4):192-193. (EI 源刊.)
  \item Yang Y, Ren T L, Zhu Y P, et al. PMUTs for handwriting recognition. In
    press. (已被 Integrated Ferroelectrics 录用. SCI 源刊.)
  \item Wu X M, Yang Y, Cai J, et al. Measurements of ferroelectric MEMS
    microphones. Integrated Ferroelectrics, 2005, 69:417-429. (SCI 收录, 检索号
    :896KM.)
  \item 贾泽, 杨轶, 陈兢, 等. 用于压电和电容微麦克风的体硅腐蚀相关研究. 压电与声
    光, 2006, 28(1):117-119. (EI 收录, 检索号:06129773469.)
  \item 伍晓明, 杨轶, 张宁欣, 等. 基于MEMS技术的集成铁电硅微麦克风. 中国集成电路, 
    2003, 53:59-61.
  \end{enumerate}

  \section*{研究成果} % 有就写,没有就删除
  \begin{enumerate}[{[}1{]}]
  \addtolength{\itemsep}{-.36\baselineskip}%
  \item 任天令, 杨轶, 朱一平, 等. 硅基铁电微声学传感器畴极化区域控制和电极连接的
    方法: 中国, CN1602118A. (中国专利公开号.)
  \item Ren T L, Yang Y, Zhu Y P, et al. Piezoelectric micro acoustic sensor
    based on ferroelectric materials: USA, No.11/215, 102. (美国发明专利申请号.)
  \end{enumerate}
\end{resume}

% 最后,需要的话还要生成附录,全文随之结束。
\appendix
\backmatter
\chapter{Rat形式语法}\label{chap:formal-syntax}

\setlength{\grammarindent}{10em}
\setlength{\grammarparsep}{5pt}
\paragraph{Syntax Rules}
\begin{grammar}
<program>        ::=    <export>+ <top level unit>*

<export>         ::=    'export' <variable>

<top level unit> ::=    <type def>
                 \alt     <variable decl>
                 \alt     <variable def>

<type def>       ::=    'newtype' <type name> '=' <constructor> <type spec>

<constructor>    ::=    <variable>

<type spec>      ::=    <primitive type>
                 \alt     <struct type>
                 \alt     <vector type>
                 \alt     <type spec> '$\to$' <type spec>

<primitive type> ::=    'Int8' | 'UInt8' | 'Int16' | 'UInt16' \alt 'Int32' | 'UInt32' | 'Int64' | 'UInt64'
                 \alt     'Float' | 'Double'

<struct type>    ::=    '\{' <variable decl> (',' <variable decl>)* '\}'

<vector type>    ::=    '[' <type name> ']'

<variable decl>  ::=    <variable> '::' <type name>

<variable def>   ::=    <variable> <variable>* '=' <expression>

<expression>     ::=    <literal>
                 \alt     <variable ref>
                 \alt     <function app>
                 \alt     <lambda exp>
                 \alt     <vector comprehension>
                 \alt     <vector element ref>
                 \alt     <vector slice ref>
                 \alt     <conditional>
                 \alt     <let exp>
                 \alt     <where exp>
                 \alt     '(' <expression> ')'

<literal>        ::=    <number>
                 \alt     <boolean>
                 \alt     <character>
                 \alt     <tuple literal>
                 \alt     <vector literal>

<tuple literal>  ::=    '(' <literal> (',' <literal>)+ ')'

<vector literal> ::=    '[' <start> ',' <end> (',' <step>)? ']'

<start>          ::=    <number>

<end>            ::=    <number>

<step>           ::=    <number>

<variable ref>   ::=    <variable>

<function app>   ::=    <function ref> <expression>*

<function ref>   ::=    <variable>

<lambda exp>     ::=    '$\backslash$' <bind var>+ '$\to$' <expression>

<bind var>       ::=    <variable>

<conditional>    ::=    'if' <test> <if clause> <else clause>?

<test>           ::=    <expression>

<if clause>      ::=    <expression>

<else clause>    ::=    <expression>

<vector comprehension>
                 ::=    '[' <expression> \\('|' <generator> (',' (<generator> | <filter>))+ ']'

<generator>      ::=    <variable> '$\gets$' <variable>

<filter>         ::=    <expression>

<vector element ref>
                 ::=    <variable> '[' <index> ']'

<index>          ::=    <expression>

<vector slice ref>
                 ::=    <variable> '[' <index> ':' <index> (',' <index>)? ']'

<let exp>        ::=    'let' <variable decl>+ <variable def>+ 'in' <expression>

<where exp>      ::=    <expression> 'where' <variable decl>+ <variable def>+
\end{grammar}

\paragraph{Lexical Rules}
\begin{grammar}
<variable>       ::=    <lower id> | <special var>

<lower id>       ::=    <lowercase> (<alpha> | <digit> | '\_')*

<special var>    ::=    '+' | '-' | '*' | '/' | '\^'

<class name>     ::=    <upper id>

<type name>      ::=    <upper id>

<upper id>       ::=    <uppercase> (<alpha> | <digit> | '_')*

<alpha>          ::=    <lowercase> | <uppercase>

<lowercase>      ::=    'a' | 'b' | ... | 'z'

<uppercase>      ::=    'A' | 'B' | ... | 'Z'

<number>         ::=    <integer>
                 \alt     <floating>

<integer>        ::=    <decimal>
                 ::=    ('0O' | '0o') <octal>
                 ::=    ('0X' | '0x') <hexadecimal>

<decimal>        ::=    <digit>+

<octal>          ::=    <octit>+

<hexadecimal>    ::=    <hexit>+

<digit>          ::=    '0' | '1' | '2' | '3' | '5' | '6' | '7' | '8' | '9'

<octit>          ::=    '0' | '1' | '2' | '3' | '5' | '6' | '7'

<hexit>          ::=    <digit> | 'A' | 'B' | 'C' | 'D' | 'E' | 'F' | 'a' | 'b' | 'c' | 'd' | 'e' | 'f'

<floating>       ::=    <decimal> '.' <decimal> <exponent>?
                 \alt     <decimal> <exponent>

<exponent>       ::=    ('e' | 'E') ('+' | '-')? <decimal>

<boolean>        ::=    'true' | 'false'

<character>      ::=    ''' <ascii> '''

<whitespace>     ::=    <space> | <tab> | <newline> | <comment>

<comment>        ::=    '--' <any character>* <newline>
\end{grammar}


\end{document}
