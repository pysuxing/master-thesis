\begin{resume}

  \section*{发表的学术论文} % 发表的和录用的合在一起

  \begin{enumerate}[{[}1{]}]
  \addtolength{\itemsep}{-.36\baselineskip}%缩小条目之间的间距,下面类似
  \item 第1作者, 第2作者, 第3作者, 第4作者, 第5作者. An abstraction for
    data-flow driven concurrent programming. Computer Science and Automation Engineering,
    2013. (EI 收录)
  %% \item Yang Y, Ren T L, Zhang L T, et al. Miniature microphone with silicon-
  %%   based ferroelectric thin films. Integrated Ferroelectrics, 2003,
  %%   52:229-235. (SCI 收录, 检索号:758FZ.)
  %% \item 杨轶, 张宁欣, 任天令, 等. 硅基铁电微声学器件中薄膜残余应力的研究. 中国机
  %%   械工程, 2005, 16(14):1289-1291. (EI 收录, 检索号:0534931 2907.)
  %% \item 杨轶, 张宁欣, 任天令, 等. 集成铁电器件中的关键工艺研究. 仪器仪表学报,
  %%   2003, 24(S4):192-193. (EI 源刊.)
  %% \item Yang Y, Ren T L, Zhu Y P, et al. PMUTs for handwriting recognition. In
  %%   press. (已被 Integrated Ferroelectrics 录用. SCI 源刊.)
  %% \item Wu X M, Yang Y, Cai J, et al. Measurements of ferroelectric MEMS
  %%   microphones. Integrated Ferroelectrics, 2005, 69:417-429. (SCI 收录, 检索号
  %%   :896KM.)
  %% \item 贾泽, 杨轶, 陈兢, 等. 用于压电和电容微麦克风的体硅腐蚀相关研究. 压电与声
  %%   光, 2006, 28(1):117-119. (EI 收录, 检索号:06129773469.)
  %% \item 伍晓明, 杨轶, 张宁欣, 等. 基于MEMS技术的集成铁电硅微麦克风. 中国集成电路, 
  %%   2003, 53:59-61.
  \end{enumerate}

  %% \section*{研究成果} % 有就写,没有就删除
  %% \begin{enumerate}[{[}1{]}]
  %% \addtolength{\itemsep}{-.36\baselineskip}%
  %% \item 任天令, 杨轶, 朱一平, 等. 硅基铁电微声学传感器畴极化区域控制和电极连接的
  %%   方法: 中国, CN1602118A. (中国专利公开号.)
  %% \item Ren T L, Yang Y, Zhu Y P, et al. Piezoelectric micro acoustic sensor
  %%   based on ferroelectric materials: USA, No.11/215, 102. (美国发明专利申请号.)
  %% \end{enumerate}
\end{resume}
