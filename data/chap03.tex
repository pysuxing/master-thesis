\chapter{Rat语法设计}\label{chap:frontend}
Rat语言语法设计的首要目标是为编程者提供易用的并行语言要素,可以简洁优雅地
描述并行问题。

Rat是一种函数式并行程序语言,它具备函数式语言的一般特性:高阶函数、恒值对象、
纯函数特性等,抽象表达能力,问题描述简洁优雅;它面向科学与工程计算领域,强化
数值计算,弱化符号计算,支持定长整数与浮点数类型,这不同于一般函数式语言;
它是一种静态强类型语言,在编译期执行类型检查,运行时安全性较强,无动态类型
检测开销;它提供了一组向量原语,可用来描述一大类数据并行操作;它被设计为一门
辅助性程序语言,为与C语言交互设计了有简单易用的接口,编程者一般只使用Rat语言编写
程序中的数据并行部分。

本章的内容结构如下:
%% 第\ref{sec:rat-overview}节对Rat语言做简单综述,
第\ref{sec:type-system}节介绍Rat的类型系统,
第\ref{sec:vector-primitives}节介绍Rat的核心并行语言要素---向量原语,
第\ref{sec:functional-characters}节对Rat语言采用的函数式语言特性进行说明,
并说明选取函数式语言设计的优势,
第\ref{sec:c-interface}节介绍Rat程序与C语言的交互界面,
第\ref{sec:n-body}节给出一个完整并行程序示例---n-body问题用以说明Rat语言的易用性。

%% \section{Rat语言综述}\label{sec:rat-overview}

\section{类型系统}\label{sec:type-system}
\subsection{Rat支持的数据类型}
在一般的程序语言中,数据与操作(函数、过程)是不同的概念,
数据就是存储在计算机当中的“数”,而操作表示施加在数据上的程序行为。
在函数式程序语言中,数据与函数的界限是模糊的,二者只不过具有不同的类型而已。
Rat是函数式语言,采用类似于Haskell的类型系统,支持数据类型与函数类型。
为了便于理解,在本节中,我们依旧区分“数据”与“函数”,下面分别予以介绍。

\subsubsection{数据类型}
Rat将数据分为两类:标量数据类型(scalar type)与向量类型(vector type)。
其中向量类型是同质(homogeneous)标量类型的一维集合。表\ref{tbl:rat-datatype}列举了
Rat的内建数据类型。

标量类型包括内建数据类型(builtin type),元组类型(tuple)。内建数据类型是计算机能够表示的基本
数据类型,包括定长整数与定长浮点数。而且,由于Rat语言面向科学计算领域,
强化数值计算能力,弱化符号计算能力。所以,Rat内建的基本数据
数据类型只包括定长整数类型与定长浮点数类型。

元组类型也称结构类型(struct),允许不同类型的元素组合得到,Rat的元组类型允许嵌套定义。

向量类型是一个同质数据集合,它包含的数据个体相同的类型,并且必须是标量类型。在Rat语言中
使用方括号表示向量类型,如\texttt{[Int32]}表示一个32位整数向量类型。
向量是一维数据集合,Rat在语法层面不支持高维数组,但通过定义元组及相应的函数可以轻松构建高维
数组程序库。向量类型是Rat语言提供数据并行能力的数据基础,这一点将在第\ref{sub:vector-operations}
中详细说明。

\begin{table}[htbp]
  \centering
  \caption{Rat数据类型}
  \label{tbl:rat-datatype}
  \begin{tabularx}{\linewidth}{|l|l|l|X|}
    \hline
    \multirow{11}*{标量类型} &
    \multirow{10}*{内建数据类型} & Int8 & 8 bit 有符号整数\\
    \cline{3-4}
    & & UInt8 & 8 bit 有符号整数\\
    \cline{3-4}
    & & Int16 & 16 bit 有符号整数\\
    \cline{3-4}
    & & UInt16 & 16 bit 有符号整数\\
    \cline{3-4}
    & & Int32 & 32 bit 有符号整数\\
    \cline{3-4}
    & & UInt32 & 32 bit 有符号整数\\
    \cline{3-4}
    & & Int64 & 64 bit 有符号整数\\
    \cline{3-4}
    & & UInt64 & 64 bit 有符号整数\\
    \cline{3-4}
    & & Float & 32 bit 单精度浮点数\\
    \cline{3-4}
    & & Double & 64 bit 双精度浮点数\\
    \cline{2-4}
    & 元组类型 & (a, b, ...) & 其中a, b是任何非函数类型\\
    \hline
    \multicolumn{2}{|l|}{向量类型} & [a] & 元素类型为a的向量类型,a为非函数类型\\
    \hline
  \end{tabularx}
\end{table}

\subsubsection{函数类型}
Rat是函数式语言,支持高阶函数等函数式语言特性,这将在第\ref{sec:functional-characters}节
详细介绍。这里只说明Rat函数的类型表示方法。

一个函数的类型由其输入参数的类型与输出结果的类型唯一决定,下面举例说明。

\begin{quotation}
  \kai{
    \texttt{dot}函数求两个单精度浮点向量的内积,它的类型声明如下:\\
    \centering\texttt{dot :: [Float] -> [Float] -> Float}\\
    其中,最后一个箭头右侧
    的类型表示dot函数返回一个\texttt{Float}类型的结果,前面两个\texttt{[Float]}表示\texttt{dot}
    函数以两个\texttt{[Float]}类型的单精度浮点向量作为输入参数。至于为什么\texttt{dot}的类型
    声明中有多于一个箭头,这与柯里化相关,这部分内容请参见第\ref{sec:functional-characters}节。
  }
\end{quotation}

\subsection{静态强类型}
Rat是一种静态强类型语言,在编译期执行严格的类型检查,保证数据类型与施加在数据上的函数类型
严格匹配。采用静态强类型的类型系统可以获得两方面优势:
\begin{itemize}
  \item 安全性:强类型检查可以保证类型安全,从而避免一大类运行时错误。
    强类型检查的内容包括,所有函数(数据)的类型皆在编译期确定;
    任何函数(数据)的类型在生命周期中保持不变;
    保证所有数据操作的有效性;不允许类型转换操作。
    Rat语言采用的类型系统类似于Haskell,而使用过Haskell
    的编程者都会有体会,那就是程序只要通过了编译,基本不会发生运行时错误。
  \item 高效率:静态类型在运行时没有类型识别与检测的额外开销,可以提高执行效率,这与Rat面向
    科学工程计算的设计初衷相符合。
\end{itemize}

\subsection{多态}
采用静态强类型系统的一个缺点就是编程灵活性受到了限制,特定类型的函数只能作用于
特定类型的数据。但本文认为,为了安全与性能承受灵活性方面的这一点代价是值得的。

为了提高编程灵活性,Rat为函数多态提供了支持。多态是指
同一种逻辑操作可以施加在多种不同类型的数据之上,如加法运算既适用于整型数,也
适用于浮点数。

多态技术可以为编程提供极大的灵活性。如果没有多态的支持,同样的加法运算就需要为不同类型
分别定义,而多态允许“一次定义,多个实例”。多态技术的典型代表是C++ template,
它定义函数或对象模板,以类型作为模板参数,使用时再根据不同的需求进行实例化。
下面的代码片段展示了使用C++ template定义的多态加法函数。
\begin{lstlisting}[language=C++]
template <typename T>
T add (T a, T b) {
  return a+b;
}
\end{lstlisting}

Rat支持多态的方式与Haskell相同,通过引入类型类(type class)的概念实现多态。
一个类型类是一组类型的集合,这些类型具有某些相同的性质,支持某些相同的操作。
如Num类型类是“数”类型的集合,他们都支持加减法运算。为了使用多态,
编程者先定义一个类型类,声明属于该类型类的类型应该支持的方法,
然后将某个类型实例化为该类型类的成员,下面的Rat代码说明了使用多态的方法。
\begin{lstlisting}[language=Haskell]
class Num a where
  + :: a -> a -> a
  - :: a -> a -> a

instance Num Int32 where
  + a b = addInt32 a b
  - a b = substractInt32 a b
\end{lstlisting}

\section{向量操作}\label{sec:vector-operations}
Rat语言的定位是数据并行语言,采用一种数据并行编程模型。数据并行问
题的特点是,将相同的操作施加到不同的数据上,不同数据操作之间的相关性较
弱,可并行度高。数据并行模型是一种层次较高的并行编程模型,编程者只需要
指出施加在数据上的操作以及该接受该操作的数据集就能实现程序的并行执行,
因此,数据并行问题的描述具有简洁优雅的特点。同时,数据并行问题涵盖了一
大类科学与工程计算问题,应用场景非常广阔。

Rat提供了若干向量操作函数,表\ref{tbl:vector-operations}中给出了一些例子。
每一个向量操作定义了一种施加在一维向量上的操作,这些操作的最大特点是,
他们在实现上是并行的。
\begin{table}[htb]
  \centering
  \caption{常用向量函数}
  \label{tbl:vector-operations}
  \begin{tabularx}{\linewidth}{p{10em}X}
    \toprule[1.5pt]
    \hei{向量原语} & \hei{功能说明} \\
    \midrule[1pt]
    map & 给定一元操作,对一个向量中所有元素施加同一操作\\
    fold & 对向量中的元素重排\\
    slice & 给定二元函数,求一个向量中元素的累加和\\
    concat & 将嵌套向量合并成一维向量\\
    zip & 将多个向量中的元素逐位置组成元组\\
    permute & 对向量中的元素重排\\
    sort & 给定排序函数,对向量中的元素进行排序\\
    \bottomrule[1.5pt]
  \end{tabularx}
\end{table}

Rat语言虽然只提供了有一维向量操作,但由于高阶函数与嵌套向量的支持,
Rat语言允许编程者实现高维数组以及表达嵌套并行。嵌套并行的例子可以
参考\ref{sec:n-body}节,第\ref{subsec:code-generation}中还要详细描述
嵌套并行的分解技术。

\section{函数式语言特性}\label{sec:functional-characters}
Rat采用了函数式语言设计,具备函数式程序语言的一般特点。在本章的后续各节中
将分别介绍Rat采用的函数式语言特性,并在最后进行总结,说明采用函数式设计带来的优势。

\subsection{函数“一等公民”}
函数式程序语言采用的计算模型是$\lambda$演算($\lambda$ Calculus)理论FIXME:citehere,是一种在功能上等价于
图灵机的计算模型FIXME:citehere。Rat是一种强类型语言,
它的理论基础是$\lambda$演算的一个扩展---类型$\lambda$演算(Typed $\lambda$ Calculus)。
在$\lambda$演算的形式化系统中,所有公式都可以看作函数,也就是说,常量被看作0元函数。
命令式语言区分“函数”与“数据”的界限在函数式语言中并不存在,函数与数据是等价的,
函数式语言中函数是一等公民,下面给出其定义:

\begin{definition}
  当程序语言中的函数具备下列性质时,称函数为该语言的“一等公民”(first-class citizen)FIXME:citehere:
  \begin{compactitem}
    \item 函数可以作为参数传递给其他函数,即支持高阶函数;
    \item 函数可以作为其他函数的返回值;
    \item 允许匿名函数
  \end{compactitem}
\end{definition}


\subsection{纯函数特性}
纯函数特性(pure functional)也称为无副作用(side-effect free)。有无副作用
描述的是程序语言中“函数”的一种性质,副作用是指,函数除了在返回某个值以外,
还改变了某些全局状态,或与程序外部的世界发生了交互。副作用,是程序语言中“函数”与数学
中“函数”的主要区别,数学函数单纯地使用输入参数计算一个返回值,而程序函数除了
计算返回值以外,还可能执行IO操作、引起异常行为等。一般来说,副作用对于程序运行
是必要的,甚至有些程序函数定义的目的就是为引起某些副作用大发生,如IO函数。

而某些程序函数,如数学函数,在执行计算的同时不引起任何副作用。我们称这类函数
为纯函数,其定义如下:
\begin{definition}
纯函数(pure function),指不引起任何副作用,
计算结果只与输入参数相关,无论何时,只要给定相同输入,就保证返回相同结果的程序函数。
\end{definition}

Rat语言面向科学工程计算设计,它关注的核心任务是数学计算,而数学计算中
使用的函数恰恰满足纯函数的特征。采用纯函数设计,不仅可以简化程序语言的
语义分析,更为程序的自动并行化提供了巨大的便利,
这将在第\ref{subsec:functional-advantages}节详细说明。

\subsection{柯里化}
柯里化(curried)也是程序函数的一种性质,它实质上是允许程序函数的形式参数
部分实例化,得到的结果仍是一个函数,这个函数的形式参数就是原函数未被实例化
的形式参数。下面仍以求解向量内积的例子说明。
\begin{quotation}
  \kai{
    \texttt{dot}函数求两个单精度浮点向量的内积,它的类型声明如下:\\
    \texttt{dot :: [Float] -> [Float] -> Float}\\
    \texttt{dot}函数有两个形式参数,当按照如下方式调用\texttt{dot}函数时,
    将可以得到两个向量\texttt{u},\texttt{v}的内积:\\
    \texttt{s = dot u v}\\
    但如果只对一个形式参数实例化,即按照如下方式调用,
    将得到一个一元函数\texttt{dotWithU},
    它有只有一个形式参数,当对这个形式参数实例化的时候,求解该实参与向量
    \texttt{u}的内积:\\
    \texttt{dotWithU = dot u}\\
    \texttt{s = dotWithU v}
  }
\end{quotation}

柯里化也可以从另一个视角理解,将所有函数都看作一元函数,一个n元函数可以看作
一个返回值为n-1元函数的一元函数。即,在多元函数的类型声明中,箭头\texttt{->}
是右结合的。这样,\texttt{dot}函数的类型声明可以看作:\\
\texttt{dot:: [Float] -> ([Float] -> [Float])}

\subsection{恒值对象}\label{subsec:immutable-object}
恒值对象也是函数式语言的重要特点。需要说明的是,这里的“对象”(object)
不同与面向对象程序语言中的对象,而是函数式语言中唯一的实体类型---函数,
之所以采用“恒值对象”而非“恒值函数”,是因为将“函数”与“赋值”联系起来
不够自然。FIXME:可以介绍传值语义以及命令式函数式混合体设计。

\begin{definition}
  恒值对象,指对象只在其定义的时候被确定其值,在其后的生命周期中不允许再次赋值,
  对象的值在整个生命周期中保持不变。
\end{definition}

Rat语言中的函数(函数与数据的总称)都是恒值对象,所以,严格来讲,Rat不支持
“变量”的概念。

恒值特性是与纯函数特性协调的,因为在现行的冯$\cdot$诺依曼计算机上,对象赋值
就代表内存区域的改写,这必然引起副作用的发生。

\subsection{函数式语言设计的优势}\label{subsec:functional-advantages}
前面介绍了Rat语言采用的若干函数式设计,下面说明为什么要采用函数式语言设计,
上述的函数式语言特性能带来什么优势。

\subsubsection{抽象层次高}
高阶函数与柯里化特性为编程提供了极大灵活性,编程者可以灵活地操作函数,
使用已有的函数构建新函数;

恒值对象将编程者从内存管理的工作中解脱出来,只需集中注意力在解决问题的逻辑
之上,内存等硬件资源的有效利用由编译期与运行时系统负责。

此外,高阶函数的使用大大增强了程序的表达能力,使得复杂的问题可以用
非常简洁优雅的代码就可以描述,且程序代码逻辑性强,可读性强。

\subsubsection{语义简单精确}
纯函数特性与恒值对象使得程序的语义更加简单精确,大大降低了语义分析的难度,
这样就使得编译器更容易掌握程序行为,同时也使得程序的高层优化更容易进行。

\subsubsection{为程序并行化提供支持}
函数式语言的理论基础是$\lambda$演算,程序的运行实质就是$lambda$演算理论中
表达式求值的过程。表达式求值的基本形式是函数应用(function application),即
通常所说的函数调用:将函数形式参数实例化,计算返回值。一个复杂表达式可能由
多个子表达式构成,各个子表达式的求值先于整个表达式的求值。

如果一个表达式中应用的函数都是纯函数,那么由$\lambda$演算理论中的
Church-Rosser定理(也称为菱形性质)FIXME:citehere可以得到下面的定理:
\begin{theorem}
  一个表达式的求值结果与子表达式的求值顺序无关。
\end{theorem}

上述定理表明,纯函数特性为程序自动并行化提供了有力支持,一个复杂表达式的
各个子表达式可以并行求值,同时保证得到正确的最终结果!

\begin{quotation}
  \kai{
    考虑下面的表达式求值问题:\\
    \texttt{dot (map f u) (scan g v)}\\
    表达式求值过程可以以一棵求值树表示,树中每一个非叶结点表示一个函数应用,
    叶结点表示函数或者数据,求值过程自下向上,最终得到根结点的值即为整个表达式的
    求值结果。上述表达式的求值树如图\ref{fig:expression-tree},其中分别以
    \texttt{t1}与\texttt{t2}为根的两棵子树代表的子表达式可以并行求值。
  }
\end{quotation}
\begin{figure}
  \centering
  \includegraphics[height=5cm]{expression-tree}
  \caption{表达式求值树}
  \label{fig:expression-tree}
\end{figure}

\section{C语言接口}\label{sec:c-interface}
正如第\ref{subsec:functional-advantages}节所说,纯函数特性是一种非常优良
的语言特性,但副作用在程序设计中也是必不可少的,因为程序要解决问题,不光
要做计算的工作,还必须与外部世界发生交互,如从外部存储器读取输入,向外部写入
计算结果等等。

如何在保留纯函数特性优势的前提下,允许副作用的发生呢?不同函数式语言有不同做法,
Lisp系语言与ML系语言不是纯函数式的,允许副作用发生,Haskell采用一种MonadFIXME:citehere
技术将程序的纯函数部分与副作用部分严格隔离开来,以便在纯函数部分仍旧能够利用
纯函数特性带来的优良性质。

Rat采用了一种不同的策略。Rat主要的设计目标是降低并行程序设计
的难度,提供易用性强的并行编程工具,由于它是一种数据并行语言,面向
科学与工程计算,所以程序的并行部分集中在计算过程,这部分程序是无副作用的。
所以,Rat被设计用来与C语言协同工作,程序的副作用部分,一般也是串行部分(如IO操作等)
使用C语言编写,计算的部分,一般也是并行性所在的部分,使用Rat语言编写。
Rat提供简单易用的C语言交互接口,允许C语言调用Rat程序。实际上,Rat编译器
就是一个源到源(source-to-source)编译器,它将Rat程序翻译成C函数供C程序调用。

Rat的C语言交互接口非常简单,仅包含一个export语句。export语句的形式语法参见附录FIXME:refhere。
通过export语句,Rat编译器将使用Rat语言定义的函数提供给C语言使用,C语言可以调用Rat函数完成
计算任务。

FIXME:有关数据类型映射的表格可以添加
%% Rat的C语言交互接口包括两个语法:import语句与export语句。
%% \begin{itemize}
%%   \item import语法形式如下:\\
%%     \texttt{import \textsl{<identifier> :: <type>}}\\
%%     该语法从外部引入一块数据,以\texttt{\textsl{identifier}}作为引用名称,
%%     其类型为\texttt{\textsl{type}}
%% \end{itemize}

\section{Rat程序实例---n-body问题}\label{sec:n-body}
本节使用Rat语言解决一个实际问题---n-body问题,以此展示Rat语言强大的抽象能力与
表达能力,同时展示Rat语言隐藏并行细节的能力。FIXME:font
\begin{quotation}
  \kai{
    n-body问题是一个经典数据并行问题。问题描述如下:

    在一个二维空间内存在$n$个粒子$[cell_1, cell_2, \cdots, cell_n]$,
    粒子具有质量$m$,位置$(P^x,P^y)$,速度$(V^x, V^y)$三种属性,
    粒子相互之间存在引力,粒子$i$与粒子$j$相互之间引力大小与他们的质量成正比,
    与距离平方成反比,即$F=G\frac{m_im_j}{D}$,
    其中$G$为引力常数,
    $D=(p^x_i-p^x_j)^2+(P^y_i-P^y_j)^2$为两个粒子间的距离。
    给定$n$个粒子的的质量$[m_1, m_2, \cdots, m_n]$,
    初始位置$[(P^x_1, P^y_1), (P^x_2, P^y_2), \cdots, (P^x_n, P^y_n)]$,
    与初始速度$[(V^x_1, V^y_1), (V^x_2, V^y_2), \cdots, (V^x_n, V^y_n)]$,
    以$t$为时间步长,求解第$n$个粒子从$0$时刻起每个时间步长之后的位置。
  }
\end{quotation}

这里给出n-body问题的Rat代码实现,并代码中的关键部分予以说明。
\begin{compactitem}
  \item 第1行的export语句表示,将\texttt{solveNBodyOneStep}函数
    输出给C程序调用,生成的C函数类型与solveNBodyOneStep在Rat语言中的类型相对应。
  \item 第3-10行定义需要使用的数据类型,这些数据类型都有对应的C语言结构类型。
    其中位置\texttt{Position}与速度\texttt{Velocity}都由一对浮点数组成,
    质量\texttt{Mass}是一个浮点数,例子\texttt{Cell}由它的三种属性构成。
  \item 第12-13行定义了引力常数\texttt{g}
  \item 第15-34行定义了完成计算功能的函数\texttt{solveNBodyOneStep},
    该函数以时间步长\texttt{t}与一组粒子\texttt{cells}为输入参数,计算\texttt{t}
    时间后粒子状态。
    \begin{compactitem}
      \item 第19-27行定义了一些辅助函数与中间变量,
        \texttt{positions}包含所有粒子的位置,\texttt{velocities}包含了所有粒子的初始速度,
        \texttt{calcDistanceSqure}计算两个粒子距离的平方和,
        \texttt{calcSingleAcc}计算单个粒子对给定粒子造成的加速度。
      \item 第28-29行定义\texttt{calcAccelerate}计算所有粒子对给定粒子造成的加速度和,
        使用了\texttt{fold}方法。
      \item 第30行定义的\texttt{accelerates}包含所有粒子在当前状态下受力产生的加速度。
      \item 第31-34行计算$t$时间后粒子的速度\texttt{newVelocities}与位置\texttt{newPositions},
        使用了\texttt{zipWith}方法。
      \item 第17行根据计算得到的$t$时间后的粒子位置与速度生成新的粒子状态,
        使用了\texttt{zipWith3}方法。
    \end{compactitem}
\end{compactitem}

\lstinputlisting[language=Haskell]{listings/nbody.hs}

观察上面的代码可以发现,使用Rat编写数据并行程序,编程者只需集中精力于问题本身的逻辑,
即书写算法,实现计算功能,程序就可以并行地运行,
而完全不用考虑内存如何分配、线程如何管理,管理并行细节的任务已经Rat编译期与运行时系统完成。
而有关Rat如何实现这一功能,将在下一章给出详细说明。

\section{小结}
本章给出了Rat语言的语法设计。

Rat是一种函数式语言,Rat采用一种类似与Haskell的语法,
但仅仅是Haskell语法一个很小的子集。
Rat的语法特点是在简洁精巧,灵活易用的同时提供了表达数据并行计算
额强大能力,这主要依靠并行向量函数。

并行向量函数从传统函数式语言中的表处理函数得到启发,
很多并行向量函数在传统函数式语言中都能找到对应的表处理函数,
但二者具有根本差别,那就是:表是一种链状结构,所以表函数是串行定义的,
而向量是数组结构,向量函数是并行执行的。

