\chapter{相关研究现状}
并行编程技术的历史不长,主要是适应并行硬件的发展在发展。但并行编程是如此重要性,
以至于自它诞生的时刻开始就一直处于学术界与工业界热点关注之下。并行编程技术种类
繁多,根据不同的标准可以做不同的归类,这些分类依据包括适用的硬件系统结构(共享内存与分布式
内存)、处理器类型(通用CPU与面向特定领域的协处理器)、并行编程模型(消息传递与数据并行)等。
本文采用并行编程工具自身设计特点作为依据,将并行编程技术分为两类:并行程序库与
并行程序语言,分别介绍两类技术的相关研究现状。由于本论文课题属于并行程序语言范畴,
将着重介绍并行程序语言的研究现状。此外,鉴于异构系统的与协处理器的重要性,还将专门
开辟一节介绍面向协处理器(主要是GPU)的并行编程技术。

\section{并行程序库技术}
并行程序库是基础的并行编程技术,一般作为系统软件提供给编程者使用。编程者可以直接
调用程序库中的API进行并行编程,也可以使用并行程序库实现更高级的并行编程工具,如
实现新的并行编程语言、开发并行编译器等。本节将介绍两类并行程序库,多线程技术应用
于共享内存计算机,消息传递接口MPI应用于分布式内存计算机。

\subsection{多线程技术}
多线程(multi-threading)技术是所有并行编程技术的实现基础,现有的并行编程技术基本
都要利用多线程技术才能实现。线程,在逻辑上是一个独立的指令序列,属于不同线程的
指令可以独立并行执行而互不干扰。多线程的概念源于多进程,同多进程一样,线程最初的设计
初衷是为了提高系统吞吐率,避免处理器因为某些耗时较大的操作(如读写磁盘)产生空转
浪费计算资源。所以,最初的多线程程序运行于单核心处理器,不同线程通过分时交替执行,
并非真正的并行执行。后来,随着多核处理器的出现,多线程程序可以真正并行地运行在
多个处理器核心之上。多线程一般由操作系统提供支持,形式为一组C语言API,实现线程的创建、
回收、通信等功能,在当前的主流通用计算平台上都有实现,如Windows的Win32线程,Linux
的LinuxThreads与NPTL。1995年IEEE发布了POSIX线程标准,称为Pthreads,Pthreads
统一了多线程程序库API,使得多线程程序具有更好的移植性。

多线程技术采用MIMD编程模型,非常适用于通用CPU的硬件结构,是对通用CPU硬件的直接抽象。
多线程技术对称序行为的控制能力强,性能高。但多线程技术抽象层次低,要求编程者显式地
控制所有并行细节,编程复杂度高,且可移植性差。总体上直接使用多线程技术的软件成本
较高。并且,只能技术应用于共享内存计算机。

\subsection{消息传递接口}
消息传递接口MPI是一个API规范,定义了一组用于在并行计算机上编写并行程序的消息传递API,
MPI也是一种编程模型,是对分布式内存计算机(尤其是由独立计算机通过网络
互联组成的计算机集群)的直接抽象。
MPI定义了运行于不同计算机的进程之间的通行行为,独立于具体程序语言与通信协议。
MPI也可以用于共享内存计算机,同样适用于定义线程间通信行为。MPI定义了上百个API,但通常
只需要使用其中6个就可以完成许多并行程序的编写。

MPI的设计初衷是高性能、可扩展性、可移植性,长期以来,MPI在分布式内存计算机系统尤其是
大型计算机集群与超级计算机上表现十分优秀,一直是在大规模科学计算领域占据支配地位的
并行编程技术。同多线程模型一样,MPI编程模型的抽象层次低,细节隐藏能力差,编程复杂度
较高。

\section{并行程序语言}
上一节介绍的两种并行程序库技术,分别是对共享内存计算机与分布式内存计算机硬件结构的直接
抽象,这两种并行编程技术对程序行为控制力强,但编程复杂度高。也就是说,并行程序库的方法
在高效性方面表现较好但在易用性方面仍显不足。

并行程序语言是解决易用性问题的根本方法,因为程序语言是人与计算机的沟通工具,抽象程度
低的语言强迫编程者使用机器的思维考虑问题,只有从语言层面提供高层的并行语法工具,才能
真正降低并行编程的难度。并行程序语言一直是并行编程技术领域的研究热点,已经有大量的
并行语言被提出,这些语言千差万别,各有适用的问题领域。但迄今为止,还没有任何一门并行语言
能够称为普遍接受的语言,适用于所有并行编程问题,多数研究成果只在特定领域应用。这是因为
当前提出的并行计算模型仍然运行在经典的冯$\cdot{}$诺依曼体系结构之上,而冯$\cdot{}$诺依曼
体系结构从根本上是串行计算模型---图灵机---的直接硬件实现。所以,当前阶段,从程序语言
层面设计并行化对语法结构,在物理实现上仍不得不回归到传统的串行计算技术。

并行程序语言的设计大致可以分为三种思路,一是设计全新的并行语言,二是扩展现有的串行语言,
三是在程序中插入编译制导指令。针对三种思路的研究均已取得众多研究成果,
下面分别介绍相关研究现状。

\subsection{并行语言设计}
本小节介绍几种并行程序语言。这些并行语言从语言本身的语法层面为程序的并行执行提供了支持,而非通过
语言扩展或者程序库的方式。本论文尤其关注函数式并行语言的研究,因为函数式语言的抽象程度高,
表达能力强,是并行编程技术易用性问题的优秀的解决方案。

\subsubsection{Multilisp}

\subsubsection{Id}

\subsubsection{Concurrent Haskell}

\subsubsection{pH}

\subsubsection{GUM}

\subsubsection{Concurrent ML}

\subsubsection{Manticore}

\subsection{串行语言扩展}

\subsection{编译器制导指令}

\section{GPU并行编程技术}

\subsection{CUDA}

\subsection{OpenCL}

\subsection{OpenACC \& OpenHMPP}
