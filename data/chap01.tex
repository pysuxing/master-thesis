\chapter{绪论}

\section{课题研究背景}

\subsection{计算机硬件系统的并行化与异构化趋势}
计算机行业从它诞生伊始就保持着日新月异的发展势头。摩尔定律(Moore's Law)FIXME:citehere的神奇预言
在过去的半个世纪中准确地反映了半导体与集成电路工业的发展速度:
集成电路上的晶体管数量每18个月翻一番。在摩尔定律的作用下,单个微处理器(microprocessor)的时钟频率
越来越高,处理器性能也随之不断提高,早期计算机性能的提高主要依赖微处理器运算
速度的提高。但在当前阶段,仅仅提高单个处理器的运算速度已经不足以使整个计算机系同的性能得到提升。

首先,单个处理器的性能提高有其极限,由于功耗问题以及晶体管器件本身物理特性的限制,单个微处理器的频率不
可能无限提高;其次,随着计算机系统结构日益复杂和多样化,访存延迟、存储器带宽等因素可能称为制约性能
的主要瓶颈;此外,程序中可发掘的指令级并行(ILP)能力有限,单一指令流难以充分利用高速处理器的处理能力。
因此,当今业界主流厂商已经将眼光投向多核处理器(multi-core processor)以及适用于特定应用的协处理器
(coprocessor)设计,希望通过为用户提供多个处理器或为特定需求提供专用处理器来获得更高的性能。

2001年IBM推出了首款通用双核(dual-core)处理器,此后,各大主流厂商相继推了不同系列的多核处理器,单个
处理器包括的核心数目不等,有的处理器核心数目可以达到16或者更多(如AMD Opteron系列,Intel Xeon系列等)。
短短几年之内,多核处理器已经称为市场的主流,从超级计算机到高性能服务器,从桌面PC到移动通信设备,多核处理器
在所有领域都得到了广泛应用。

协处理器一般是针对特定应用而设计的处理器,包括面向浮点计算、图形处理、信号处理、加密解密等。相较于通用
处理器,协处理器可能缺乏某些功能,但在特定的功能上具备突出的性能优势。许多计算机系统都配备了一个或
多个协处理器,用以辅助CPU执行计算。图形处理器(Graphics Processor Unit, GPU)是当前应用最为广泛的一类协处理器,
它最开始被专门设计于图形处理问题,协助CPU完成图形渲染等工作,但现阶段GPU已经越来越成为一种“通用”计算设备,
从软硬件两方面都为通用计算提供支持。GPU采用流处理(stream processing)编程模型,这是一种类似于SIMD(Single
Instruction Multiple Data)的编程模型。在流处理模型中,一组数据中的每一个个体都被
施加同一操作,一般将这种操作称之为Kernel。Kernel可以是一段程序,Kernel允许有自己的逻辑,而不限于单一指令,
这比SIMD提供了更高的灵活性。在硬件实现上,GPU拥有成百上千个核心,Kernel通常被流水化执行,这可以大大加速
数据级并行问题的执行速度。

除使用多核处理器以外,由通用处理器与协处理器组成异构硬件系统(heterogeneous architecture)
也已经成为当前计算机系统的主流构成模式。由国防科学技术大学研制的,在2013年度Top500FIXME:citehere评选中
排名第一的天河二号超级计算机,就采用了这种异构系统设计,它的每个计算结点由两颗Xeon E5 12核处理器(CPU)
与三颗Xeon Phi 61核协处理器(GPU)组成,CPU+GPU的硬件配置为大规模科学计算提供了强大的计算能力。在桌面
计算机市场上,一般PC机都配备了专门的图形处理器,这些图形处理期多数由Nidia与ATI公司生产,除提高计算机在
游戏方面的处理能力之外,也为通用计算提供软件支持。最新的智能手机也已经开始采用GPU来辅助CPU执行计算,
以期为游戏提供更强的计算支持。

总之,并行化与异构化是当今计算机硬件系统的主流发展趋势,在可以预见的将来,计算机性能的提高将以并行能力
的驱动与异构硬件的利用为核心任务。

\subsection{并行化与异构化硬件的软件支持}
多核处理器与协处理器为设计实现更高性能的计算机提供了硬件支持,如何适应
硬件的发展,有效地利用这些硬件资源,为程序员提供易用的、高效的编程工具,是提高计算机应用能力
的关键问题。并行编程工具的设计需要兼顾两个核心特性:FIXME:emphasizehere有效性与易用性。有效性是指使用编程工具
必须能偶充分开发并行硬件的计算能力,最大程度地利用硬件资源;易用性是指在有效开发硬件并行能力的前提下,
编程工具应该是易用的,并行程序设计的难度与复杂度不能过高。已经有多种并行编程工具得到了广泛使用,这些工具有的
针对多核处理器设计,有的针对异构硬件设计,有的适用于数据并行问题,有的适用于任务并行问题。这些并行编程
工具包括多线程、消息传递接口、并行程序语言、编译器制导指令等,它们特点各异,在通用性、适用问题方面
都有不同,下面将简要介绍。

多线程(multithreading)技术是在传统串行编程技术的基础上发展起来一种编程模型,
它的出现早于多核处理器。一个线程在逻辑上是一条指令流,不同线程的指令流可以在单一处理器上交替
执行以隐藏一些耗时较高的操作(如IO),避免处理器空转,提高处理器的吞吐率。
在多核处理器上,多个线程可以真正“并行”地执行。使用多线程编程模型,编程者显式地
将整个任务划分为独立的子任务,不同的子任务在不同的线程中执行,从而利用更多处理器资源,
缩短整个程序的运行时间。多线程支持通常由操作系统(Operating System)提供,是一种非常成熟的并行
编程技术,POSIXFIXME:citehere线程是多线程模型的POSIX标准,Pthreads在多种操作系统上均有实现。

消息传递接口(Message Passing Interface, MPIFIXME:citehere)是一套并行程序库接口规范,它以多线程技术为基础,定义了线程间
通信的标准方法,最初定位用于分布式内存计算机系统,但也适用于其他体系结构的计算机系统。MPI定义的标准
通信接口功能全面,问题描述能力强大,而且具被性能高、可扩展性强、可移植性强的特点,长期以来一直都是
性能计算领域的主要编程模型。

多线程技术与MPI技术都是在已有的串行程序语言之上,通过构建程序库的方式为并行编程提供支持。这种方式的
好处是硬件控制能力强,执行效率高,缺点是受限于已有的编程语言功能,细节隐藏能力差,编程复杂度高。
并行程序语言采用一种不同的思路,从语言层面为并行程序提供支持。并行程序语言一般通过提供特殊的语法
结构来表达程序中的并行部分,但在底层实现上仍采用多线程与MPI技术,任务并行化的工作由编译器完成。
并行编程语言的抽象层次一般较高,更加注重语言的易用性,对编程者隐藏硬件细节。并行程序语言一直
是并行软件技术的研究热点,这部分内容将在下一章着重介绍。

编译器制导指令(Compiler Directive)技术的允许程序员在源代码中插入专用的制导指令(directive)来指出
程序中的并行部分,并在必要之处加入同步互斥及通信,编译器识别这些制导指令,对程序做自动并行化处理。
编译期也可以视情况忽略这些制导指令,这时程序退化为串行程序。这种技术只需要程序员做简单的制导工作,
同时具有较好的可移植性,可以根据硬件资源的数量自动调整并行度,缺点在于难以调试,缺乏错误处理,对线程
粒度控制较弱,同时很难应用与非共享内存计算机系统。编译器制导指令技术的代表是OpenMPFIXME:citehere与OpenACC
FIXME:citehere。

开发协处理器并行能力的技术在学术界与工业界都受到高度关注,发展也十分迅速。Nvidia公司针对自己的GPU
首先提出了CUDA编程架构,使用CUDA C语言对GPU进行编程,CUDA C是一种C语言的扩展语言FIXME:citehere。著名的
非盈利技术联盟Khronos Group针对异构计算提出了OpenCL标准。编译器制导指令技术在协处理器上也得到了应用,代表
有OpenACC与OpenHMPP。

总体上,并行程序设计的难度远远高于串行程序设计。在基于传统串行程序语言的解决方案中,编程者不仅需要
关注于解决问题的程序逻辑,还需要显式地处理各种并行细节,诸如线程创建、内存管理、消息传递、任务划分等。
并行程序语言设计从根本上解决了编程难的问题,但现有的实现技术仍有很大改进空间,很多研究成果只能适用于
特定领域特定问题,通用性不强。相较于并行硬件的发展,并行软件技术的发展现状仍然不够理想,现阶段
并行软件技术还没有一个完美的解决方案。有效性与易用性的平衡与兼顾仍然是一个亟待解决的关键问题。

\section{课题研究意义}
正如上一小节所说,计算机硬件系统结构的主流发展趋势是并行化与异构化,而开发并行化与异构化硬件的计算能力
需要相应的软件技术提供并行编程工具。未来计算机的性能提高,关键就在于如何从软硬件两方面开发计算机系统
的并行执行能力。这其中,软件要适应硬件的发展,根据硬件的特性设计软件界面,在保证有效性的前提下尽量兼顾
易用性。具体来说,软件工具需要解决的问题有下列三点,其中前两点是实现并行软件的有效性要求,第三点是
易用性要求。
\begin{itemize}
  \item 如何最大限度地开发并行硬件(包括多核通用处理器与众核协处理器。)的计算能力?
  \item 如何设计编程模型,使异构系统中通用处理器与协处理器更加有效地协调配合?
  \item 如何对用户隐藏计算机系统的并行硬件细节,降低并行程序设计的难度?
\end{itemize}

本论文着眼于上述并行软件需要解决的三个问题,对并行程序语言展开研究,借鉴函数式程序语言的优良性质,
设计了一门数据并行程序语言Rat,为编程者提供一个抽象层次高、表达能力强、细节隐藏好的编程界面,大大降低
并行程序设计的难度。并行程序语言从根本上解决并行编程难的问题,对降低并行计算机系统上的应用开发难度、
提高并行计算机可用性具有重要意义。

%% Rat语言可以在GPU上高效实现,可以有效地开发异构硬件系统的并行硬件,并且可以
%% 根据硬件资源的情况自动调整执行策略,因此,使用Rat语言编写的并行程序能够在不同的硬件配置下
%% 均达到较高的性能,同时具备较强的可移植性。

\section{主要研究内容}
函数式并行程序语言Rat的研究内容主要分为两方面:前端语法设计与后端编译实现技术。其中前端语法设计着重实现
并行编程工具的易用性,后端编译实现技术着重解决并行编程工具的有效性问题。

\subsection{并行程序语言语法设计}
Rat语言的语法设计旨在为编程者提供一个通用、易用的并行编程模型,Rat语言具备以下优良特性:
\begin{itemize}
  \item 抽象层次高。编程者在编程解决问题时只需要关注问题本身的逻辑,无需关注底层的并行实现细节,线程管理、
    内存管理、任务划分等工作由编译器与运行时系统执行。
  \item 表达能力强。Rat精心选取了一组并行原语,使用这一组有限的并行原语就可以方便地描述一大类数据并行问题。
  \item 语法简洁精巧,易学易用。
\end{itemize}

\subsection{并行程序语言编译实现技术}
Rat语言的实现目标是兼顾易用性与有效性。易用性由它的语法设计提供,有效性有赖于高效的编译实现技术。
本论文在Rat语言的编译实现技术方面主要研究点包括:
\begin{itemize}
  \item 并行虚拟机的设计。并行虚拟机是对并行硬件的抽象,它运行精简的指令集,易于分析和优化,并且在
    并行硬件上有高效实现。
  \item 开发协处理器的并行计算能力。Rat的首个实现采用Nvidia公司的GPU作为并行计算硬件,作为并行虚拟机
    的实现平台。在编译实现过程中着重考虑如何充分利用GPU的流处理器资源、共享存储器资源,
    如何达到更高的访存带宽等。
  \item 开发异构硬件的系统工作能力。借鉴函数式程序语言的优良特性,对如何使CPU与GPU协同、并行工作展开研究,
    使不同的处理器资源发挥各自所长,提高系统整体的性能。
\end{itemize}

\section{论文结构组织}
本论文设计并实现了一门并行程序语言Rat。全文组织结构如下:

第一章介绍了论文的研究背景,指出计算机硬件并行化与异构化的发展趋势,概括了并行软件技术的发展目标,
分析了当前并行软件技术的不足,阐述了本课题的研究意义,简述了本课题的主要研究内容。

第二章介绍了并行软件技术领域的国内外研究现状,分析现有研究采用的方法,取得的成果以及存在的不足之处。

第三章详细说明了Rat语言的语法设计,分别就函数式语言特性、类型系统、向量原语展开描述。

第四章说明Rat语言编译实现所采用的关键技术,包括并行虚拟机的设计、并行虚拟机在并行硬件上
的实现与优化、异构硬件系统的协同工作技术研究与并行自动化技术研究等方面。

第五章FIXME:shiyan

第六章对论文的工作进行总结,指出了论文的不足以及未来的工作方向。

%% \begin{figure}[htp]
%% \centering
%% \includegraphics{picmain}
%% \caption{图 1.1 名称}
%% \end{figure}

%% \begin{table}[htp]
%% \centering
%% \caption{表 1.2 名称}
%% \begin{tabular}{|c|c|c|c|c|}
%% \hline
%% \makebox[2.07cm][0pt]{} & \makebox[2.07cm][0pt]{} & \makebox[2.07cm][0pt]{} & \makebox[2.07cm][0pt]{} & \makebox[2.07cm][0pt]{} \\
%% \hline
%%  & & & & \\
%% \hline
%%  & & & & \\
%% \hline
%% \end{tabular}
%% \end{table}

