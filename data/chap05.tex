\chapter{Rat运行时系统}
第\ref{chap:compiler}章介绍了Rat的编译技术,Rat编译器生成的C程序实质上是
对运行时系统的API调用。本章将详述Rat运行时系统在CPU/GPU异构硬件系统上的设计与实现。
运行时系统结构如图\ref{fig:backend}。
\begin{figure}
  \centering
  \includegraphics[width=\linewidth]{backend}
  \caption{运行时系统结构图}
  \label{fig:backend}
\end{figure}

从图\ref{fig:backend}中可以看出,运行时系统由三部分组成:
内存资源管理器(Memory Manager)、并行任务调度器(Task Scheduler)与向量原语驱动(VP Driver)。
第\ref{sec:memory-manager}节介绍内存资源管理器,该组件对整个系统中可用的内存资源实施统一的管理。
第\ref{sec:task-scheduler}节介绍并行任务调度器,该组件的输入为编译期静态生成的任务队列,
在程序运行期动态地将队列中的计算任务发送到GPU上,同时在CPU端协同GPU完成一些辅助计算。
%% 第\ref{sec:vp-driver}节介绍
向量原语驱动就是向量原语具体硬件上实现,这里采用Nvidia GPU作为硬件平台,
Core语言的向量原语在GPU的实现技术已经比较成熟,如广泛被使用的CUDA程序库CUDPP\upcite{Harris2007,Blelloch1990},
这里就不再介绍。

\section{内存资源管理器}\label{sec:memory-manager}
由于Rat运行在CPU/GPU异构硬件系统中,向量操作在GPU端完成,所以向量的数据必须存储在GPU端内存中。
GPU上的内存分配与回收必须在CPU端通过特定的API调用完成,
即,虽然内存资源本身位于GPU端,但GPU内存分配与回收的管理功能却要在CPU端执行。
这就造成了向量对象的数据内容存储在GPU端内存、元信息(长度、元素类型等)存储在CPU内存的特殊内存布局,
图\ref{fig:vector-memory}表示了向量对象这一特点。
%% 由于这种特殊的内存分布方式,大部分Core语言指令都在GPU端执行,但也存在某些指令
%% 在CPU端执行,如表\ref{tbl:assist-instruction}中列出的\texttt{length}指令,
%% 以及向量操作内部的\texttt{if}指令等。
\begin{figure}
  \centering
  \includegraphics[height=5cm]{vector-memory}
  \caption[向量内存分布]{向量内存在CPU/GPU硬件系统上的分布}
  \label{fig:vector-memory}
\end{figure}

内存资源管理器的功能是,管理CPU端与GPU端的内存空间,对并行任务调度器的内存请求产生响应。
它主要管理两类内存空间:GPU全局内存(global memory)与CPU端页锁内存(page-locked memory)。

%% 一般来说,CPU端的内存管理可以直接调用操作系统API完成,
%% 有一类特殊的CPU内存——页锁内存需要特殊考虑。
\subsection{GPU全局内存管理}
由于Rat程序中包含大量的向量创建与回收操作,而GPU驱动提供的内存分配与回收API
不仅开销巨大,而且可能会中断GPU上正在执行的计算任务,所以,运行时系统内部
实现了一个GPU全局内存管理器,维护程序中使用的GPU存储空间,处理程序运行期
对GPU内存的分配与回收。

第\ref{subsec:immutable-object}节已经指出,Rat语言中的所有对象在整个
生命周期中是不可改写的,称为恒值对象。也就是说,每当一个数据操作(标量操作
或向量操作)被施加到一个对象上时,逻辑上都会产生一个新的对象,编程者可以认为
新的对象与原有对象使用不同的内存空间。

恒值对象简化了程序语义,但如果在具体实现中,
为每一个对象都分配新的内存空间显然是不经济的,
在Rat的实现中,采用了三种内存空间优化技术,可以在
保持程序高层语义简洁的同时提高内存空间利用率。
这三种技术分别是:向量内存即时回收(vector memory collection)、
零拷贝(zero-copy)、向量原语原地执行(execute in-place)。

下面先给出GPU全局内存管理的基本方案,然后分别介绍三种内存优化技术。

\subsubsection{GPU全局内存管理器}
内存管理器在初始化阶段一次性申请一大块连续内存空间作为可用内存池,
其大小通过程序参数指定。可用内存池采用一种特殊的双向链表组织,
结构如图\ref{fig:gpu-memory-manager}。每一个可用内存块都在自己
自己的头部存储本块的大小、链表中前一个与后一个可用内存块的地址,
内存管理器只需要知道第一块的地址,就可以获取所有可用内存的信息。
\begin{figure}
  \centering
  \includegraphics[width=\linewidth]{gpu-memory-manager}
  \caption{GPU内存管理器}
  \label{fig:gpu-memory-manager}
\end{figure}

内存分配算法采用最佳匹配算法(best-fit)。这基于Rat程序的一个特点,那就是
处于同一条流上的向量对象很可能有相同的大小,也就是说,可用内存块链表中
很有可能存在与分配请求刚高匹配的内存块,这样就避免了碎片的产生。

内存回收算法采取合并式回收,即如果在物理区域上与被释放的块相连的内存块可用,
则将他们合并为一个大的可用内存块。

\subsubsection{向量内存即时回收}
首先介绍向量内存即时回收技术。一段程序中往往存在许多的中间对象,
他们在整个生命周期中只被引用一次,作用相当于临时变量。
当对一个对象的最后一次引用结束之后,即使该对象的生命周期还没结束,
但它的内存空间已经可以被回收利用而不影响程序的正确性。也就是说,
在物理实现中,对象占用分配给它的内存空间的时间可以短于它的逻辑生命周期。
重用内存空间意义重大,对内存空间开销很大的长向量数据执行内存回收重用尤其重要。
%% 图\ref{fig:object-lifetime}给出了一个例子说明。
\begin{quotation}
    \kai{
    以\ref{subsec:functional-advantages}节中表达式求值问题为例:\\
    \centerline{\texttt{dot (map f u) (scan g v)}}\\
    上述表达式的求值树参见图\ref{fig:expression-tree}。整个计算过程涉及5个向量数据,
    分别是两个输入向量\texttt{u}与\texttt{v},两个中间向量\texttt{t1}与\texttt{t2},
    最终结果向量\texttt{result}。其中\texttt{u}与\texttt{v}分别
    在计算\texttt{t1}与\texttt{t2}的时候被引用一次,之后不再被引用,\texttt{t1}与
    \texttt{t2}在计算最终结果\texttt{result}的时候被引用一次,之后不再被引用。
    5个向量的的逻辑生命周期见图\ref{fig:object-lifetime},图中还给出了各向量
    引用计数随时间的变化情况(以红色标识)以及他们实际占用内存空间的时间区间(以蓝色标识)。
    
    从图\ref{fig:object-lifetime}可以看出,\texttt{result}与\texttt{u},\texttt{v}的
    内存空间占用时间区间不存在重合,且在\texttt{result}实际占用内存空间的时间段内
    \texttt{u}、\texttt{v}的引用计数都为0。在这种情况下,\texttt{u}、\texttt{v}占用
    的内存空间就可以回收由\texttt{result}重用。
  }
\end{quotation}
\begin{figure}
  \centering
  \includegraphics[scale=0.8]{object-lifetime}
  \caption{对象生命周期与内存空间占用示意图}
  \label{fig:object-lifetime}
\end{figure}

\subsubsection{零拷贝}
零拷贝也称为写时拷贝(copy-on-write),指某些向量操作虽然逻辑上定义了新的向量,
但实际上没有对输入向量进行任何处理,
得到的新向量与原向量内容相同或部分相同,这时新向量就可以直接指向输入向量的内存空间。
这时,逻辑上可能有多个向量指向同一块内存空间。因为前面提到的两种内存优化技术可能
需要改写向量内存空间(指向该空间的某个向量引用计数变为0或有向量原语试图原地执行),
这时视开销情况执行内存拷贝动作或是为改写动作分配新内存。

向量原语对零拷贝的支持情况参见表\ref{tbl:copy-on-write}。
\begin{table}
  \centering
  \caption{向量原语的零拷贝支持}\label{tbl:copy-on-write}
  \begin{tabularx}{\linewidth}{ZZZ}
    \toprule[1.5pt]
    \hei{向量原语} & \hei{是否支持零拷贝} & \hei{备注}\\
    \midrule[1pt]
    map & $\times$ &\\
    scan & $\times$ &\\
    gpermute & $\times$ &\\
    gcopy & $\surd$ & 拷贝步长必须为1\\
    shrink & $\times$ &\\
    sort & $\times$ &\\
    random & $\times$ &\\
    generate & $\times$ &\\
    zip & $\surd$ &\\
    \bottomrule[1.5pt]
  \end{tabularx}
\end{table}

\subsubsection{向量原语原地执行}
有些向量原语可以原地执行。所谓原地执行,就是向量操作直接在输入向量的内存空间上
写入结果,不需要申请新的内存空间。原地执行必须满足一个条件,就是输入向量在本次向量操作之后引用计数变为0,
即在本次向量操作中输入向量是最后一次被引用。
\begin{quotation}
  \kai{
    在图\ref{fig:object-lifetime}所示的生命周期中与内存空间占用情况中,虽然\texttt{t1}与\texttt{u}
    的内存空间占用时间区间存在重合,但由于\texttt{u}在\texttt{map}操作中是最后一次被引用,
    同时,\texttt{map}原语支持原地执行,这时,\texttt{t1}可以直接回收利用\texttt{u}的内存空间。
  }
\end{quotation}

表\ref{tbl:inplace-vp}给出了向量原语对原地执行的支持状况,某些向量原语需要同步操作才能完成原地执行。
\begin{table}
  \centering
  \caption{向量原语的原地执行支持}\label{tbl:inplace-vp}
  \begin{tabularx}{\linewidth}{ZZZ}
    \toprule[1.5pt]
    \hei{向量原语} & \hei{是否支持原地执行} & \hei{是否需要同步}\\
    \midrule[1pt]
    map & $\surd$ & $\times$\\
    scan & $\times$ & --\\
    gpermute & $\surd$ & $\surd$\\
    gcopy & $\times$ & --\\
    shrink & $\times$ & --\\
    sort & $\times$ & --\\
    random & $\surd$ & $\times$\\
    generate & $\surd$ & $\times$\\
    zip & $\times$ & --\\
    \bottomrule[1.5pt]
  \end{tabularx}
\end{table}


%% \subsubsection{GPU全局内存管理算法}
%% 前面介绍了三种内存空间优化技术,GPU全局内存管理器使用这三种技术管理GPU全局内存
%% 的分配与回收。

%% 算法\ref{alg:gpu-memory-manager}给出了GPU全局内存分配算法
%% 的正规描述。该算法接收一个任务作为输入,为该任务的输出分配内存。
%% 该算法首先检查任务的向量操作是否能够零拷贝执行,如果可以,则直接将任务输入向量的引用计数
%% 增一;若任务的向量操作可以原地执行,而且任务的输入向量引用计数为1,则原地执行;否则,
%% 调用内存分配器分配内存;内存分配器现在空闲内存表中搜索可用内存,如果搜索不到,则退化为调用
%% GPU提供的内存分配API。
%% \begin{algorithm}
%%   \caption{GPU全局内存分配算法}
%%   \label{alg:gpu-memory-manager}
%%   \begin{algorithmic}[1]
%%     \Require 任务$task$
%%     \Ensure 设置用于存放$t$的结果向量的内存区域$mem$。

%%     \Function{task-mem-alloc}{$task$}
%%     \State $vop \leftarrow get\_vop(task)$
%%     \State $mem \leftarrow NULL$
%%     \State $input \leftarrow get\_input(task)$
%%     \If{$zero\_copy(vop) = TRUE$}
%%     \State \Call{inc-mem-ref}{$input$}
%%     \State $mem \leftarrow input$
%%     \ElsIf{$exc\_inplace(vop) = TRUE \& mem\_ref(input) = 1$}
%%     \State $mem \leftarrow input$
%%     \Else
%%     \State $mem \leftarrow$ \Call{mem-alloc}{$get\_output\_size(task)$}
%%     \EndIf
%%     \State $set\_output(mem)$
%%     \EndFunction

%%     \Function{mem-alloc}{$size$}
%%     \State $mem \leftarrow$ \Call{search-free-memblock}{$size$}
%%     \If{$mem = NULL$}
%%     \State $mem \leftarrow$ \Call{gpu-malloc}{$size$}
%%     \EndIf
%%     \State \Return{mem}
%%     \EndFunction
%%   \end{algorithmic}
%% \end{algorithm}

%% 对于已经执行完毕的任务,要检查他们的内存是否能够回收。
%% 算法\ref{alg:gpu-memory-collect}给出了GPU全局内存回收算法的正规描述。
%% 该算法接受一个已经完成的任务作为输入,如果该任务不是零拷贝或原地执行操作,
%% 则检查它的输入内存引用计数是否为0,为0则将内存区域添加到可用内存队列。
%% 最后检查可用内存队列的大小,如果超过某个阈值则调用GPU API释放一些内存。
%% \begin{algorithm}
%%   \caption{GPU全局内存回收算法}
%%   \label{alg:gpu-memory-collect}
%%   \begin{algorithmic}[1]
%%     \Require 执行完毕的任务$task$
%%     \Ensure 对$task$占用的内存执行回收
%%     \Function{task-mem-free}{$task$}
%%     \State $vop \leftarrow get\_vop(task)$
%%     \State $input \leftarrow get\_input(task)$
%%     \If{$zero\_copy(vop) = FALSE \& exc\_inplace(vop) = FALSE$}
%%     \If{$mem\_ref(input) = 0$}
%%     \State \Call{mem-free}{$mem$}
%%     \EndIf
%%     \EndIf
%%     \If{$free\_list\_size() > THREADHOLD$}
%%     \State \Call{release-gpu-mem}{}
%%     \EndIf
%%     \EndFunction
%%   \end{algorithmic}
%% \end{algorithm}

\subsection{CPU页锁内存管理}
在CPU端使用页锁内存可以提供一些性能上的优势\upcite{Nvidia2008}
GPU可以在执行Kernel的同时进行页锁内存与GPU内存之间的数据传输,
某些GPU设备可以直接读取页锁内存从而避免CPU端与GPU端的数据传输。
但页锁内存不能被操作系统用来响应正常的页请求,申请过多的页锁内存会导致操作系统性能下降。

基于页锁内存的上述性质,首先要确定一个页锁内存的上限,在程序的运行过程中
从作系统处申请的页锁内存不能超过该限制。然后要考虑在何种情况下使用页锁内存
能够提高性能。具体来说,当需要执行的任务满足下面的条件时使用页锁内存:
\begin{compactitem}
  \item 任务需要在CPU与GPU之间传输数据
  \item 需要传输的数据块只被读写一次
\end{compactitem}

不同于GPU全局内存的管理,
页锁内存采用即时分配即时释放的策略,需要使用的时候向操作系统申请,
一旦使用完毕则立即归还操作系统,这是为了避免占用过多页锁内存影响操作系统性能。

\section{并行任务调度器}\label{sec:task-scheduler}
并行任务调度器的功能是在程序运行期将编译期生成的任务动态地调度到GPU上运行。
想要使程序的运行更加高效,任务的动态调度必须实现下面两个目标:
\begin{compactitem}
  \item 最大化地利用GPU硬件资源
  \item 彼此独立的任务能够在硬件资源允许的情况下并行执行
\end{compactitem}

第\ref{cpu-gpu-model}节先简要介绍CPU/GPU异构硬件系统的编程模型,
Rat运行时系统的动态调度技术即针对CPU/GPU异构硬件系统的特点设计,
这部分内容将在第\ref{subsec:task-scheduling}节给出。

\subsection{CPU/GPU并行编程模型}\label{cpu-gpu-model}
在CPU由GPU组成的异构硬件系统中,一般将CPU端与GPU端分别称为主机端(host)与设备端(device)。
主机端与设备端各自有独立的内存地址空间,主机内存与设备内存一般通过一根PCIe总线连接。

GPU拥有多个流处理器(multi-processor),每个流处理器拥有数十乃至上百个处理器核,可以同时
运行大量线程。GPU一般作为协处理器协助CPU完成计算密集的任务,发送到GPU上运行的任务称为Kernel,
CPU端线程通过API调用将Kernel发送到GPU上执行。

GPU采用层次化设计组织线程,每个线程grid包含若干线程block,每个线程block包含若干线程,
grid与block可以按照一维、二维、三维数组的方式组织。其中,线程block是GPU执行任务调度
的基本单元,所以,如果一个线程block的规模过大超过了某些资源限制(如可用寄存器数量),
可能会造成任务失败。

在GPU端,在任意时刻只能有一个活跃的CPU线程对GPU发送命令,称该CPU线程称为上下文(context)。
CPU与GPU可以交替执行任务,这种情况下,CPU线程将Kernel发送到GPU,然后阻塞等待GPU完成计算返回结果。
CPU也可以采用异步方式发送Kernel,在GPU执行计算工作的同时可以并行处理其他任务。
显然,从提高系统综合性能的角度来讲,应该尽量使CPU与GPU并行工作。

GPU上可以重叠执行多个Kernel,也可重叠执行Kernel与主机端设备端数据传输,
这通过GPU流来实现。被发送到不同的GPU任务流的任务(包括数据传输与Kernel计算)在硬件
资源允许的情况下可以同时执行。

\subsection{并行任务动态调度}\label{subsec:task-scheduling}
根据CPU/GPU异构硬件系统的特点,我们为并行任务调度器设计了调度算法,该算法以流探测算法
为基础,从程序的向量流图中搜索“单行流”,并以单行流作为任务动态调度的单元。

\subsubsection{单行流}
首先给出单行流的定义。
\begin{definition}
  向量流图中,从一个结点$n_0$出发沿某条流向前经过$k$个结点$n_1, n_2, \cdots, n_k$,
  若$n_k$是分支结点或流的终点,或沿流的方向向前的下一个结点$n_{k+1}$是汇合结点,
  称由$n_0, n_1, \cdots, n_k$与连接他们的边组成流路径为单行流。
\end{definition}

单行流的特点是,单行流上所有数据结点依赖且仅依赖该单行流上位于它前面的结点。
单行流上的操作必须顺序执行,不可能并行执行,
同时,两条单行流可以完全独立地执行,他们的结点之间不存在任何依赖关系。
单行流的这种特点与GPU任务流的特点完全相同,所以,以单行流为单位,将一条单行流上
发送到到一条GPU任务流上执行是非常合适的。

\begin{quotation}
  \kai{
    观察n-body问题的向量流图\ref{fig:n-body-stream},从\texttt{cells}出发的单行流共有
    四条,分别是:\\
    \centerline{\texttt{cells -> masses}}\\
    \centerline{\texttt{cells -> t4 -> calcSingleAcc}}\\
    \centerline{\texttt{cells -> velocities}}\\
    \centerline{\texttt{cells -> positions}}\\
    这四条流代表的任务就可以同时被发送到GPU上执行。
    注意从\texttt{cells}指向\texttt{accelerates}的\texttt{map}操作已经被
    转换为\texttt{foreach}命令。
  }
\end{quotation}

由于在编译期任务生成的时候为运行时系统保留了向量流图的依赖关系,
所以运行时系统可以根据流图搜索可用的单行流将之发送到GPU任务流上。
下面给出任务调度器的动态调度算法。

\subsubsection{并行任务动态调度算法}
任务调度器维护下面几个数据结构:GPU任务流队列、可调度单行流队列、
待调度任务集合、新就绪对象集合。

在程序开始执行的时候,GPU任务流初始化若干条任务流,任务流的数目在程序运行期不再变化,
然后任务调度器开始搜索可调度单行流发往相应的任务流。

在程序开始执行的时候,
任务调度器首先以输入向量对象为起点开始搜索单行流,将搜索得到的所有可用单行流
加入可调度单行流队列,然后将该队列中的单行流发送到空闲的GPU任务流。
当所有GPU任务流都被占用或者可调度单行流队列已经排空时,
任务调度器对各个GPU任务流进行轮询,
直至某条GPU任务流上的单行流执行完毕。执行完毕的单行流会造成某些向量对象
的从数据为就绪状态转入就绪状态,这时调度器从这些新增的就绪向量对象开始
搜索单行流加入可调度单行流队列,开始新一轮调度。

算法\ref{alg:task-scheduling}给出了并行任务动态调度算法的正规描述。
\begin{algorithm}
  \caption{并行任务动态调度算法}
  \label{alg:task-scheduling}
  \begin{algorithmic}[1]
    \Require GPU任务流队列$GPUQueue$,可调度单行流队列$SSQueue$,新就绪对象集合$vectors$
    \Function{schedule}{}
    \Loop
    \For{$v$ in $vectors$}
    \State $ss \leftarrow$ \Call{search-sstream}{v}
    \State $SSQueue \leftarrow SSQueue \cup \left\{ ss \right\}$
    \EndFor
    \If{$empty(SSQueue) = TRUE\ or\ allbusy(GPUQueue) = TRUE$}
    \For{$gq$ in $GPUQueue$}
    \If{$finished(gq)$}
    \State $vectors \leftarrow ready\_vectors(gq)$
    \State \textbf{goto loop} 
    \EndIf
    \EndFor
    \EndIf
    \EndLoop
    \EndFunction
  \end{algorithmic}
\end{algorithm}

%% \section{向量原语的GPU实现}\label{sec:vp-driver}
%% %% 同时还根据GPU本身的硬件限制(如流处理器数量、单个流处理器线程上限、单个流处理器寄存器综述等)
%% 表\ref{tbl:vector-primitives}中列出的向量原语中,\texttt{zip}原语支持零拷贝,
%% 可以在CPU端以常数时间完成,其余的向量原语在GPU上都有高效实现。

%% CUDPP

\section{小结}
本章对Rat的运行时系统设计作出了详细描述。
运行时系统提出了两点创新性设计:

在CPU端实现了一个GPU全局存储器管理器,代替编程者完成GPU端的内存自动分配回收任务。
该内存管理器采用了三种优化策略减少程序执行过程中的内存开销。

设计了基于单行流的并行任务动态调度算法。单行流的概念与GPU任务流的性质相似,
使用单行流作为任务调度单元非常适合现有的GPU软硬件结构,同时可以
保证不同任务之间并行执行的能力。

