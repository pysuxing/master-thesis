\chapter{Rat编译实现技术}\label{chap:compiler}
第\ref{chap:frontend}章介绍了Rat语言的语法设计,简洁优雅的语法为编程者提供了
良好的编程界面,解决了并行编程工具的易用性问题。那么另一方面,
高效性要依靠语言的编译实现技术解决。

本章将详细说明Rat语言编译实现的关键技术,包括中间语言Core的设计,
流驱动并行计算模型的使用与流探测技术,两种流优化算法以及最终的代码生成方案。
本章内容结构安排如下:
第\ref{sec:compiler-overview}节介绍Rat语言编译实现的总体方案,总览Rat程序的编译流程。
第\ref{sec:core-language}节给出了Core语言的设计,Core语言为Rat提供了数值计算能力与
数据并行能力,它提供的向量原语是可以在多种并行硬件上高效实现的。
第\ref{sec:stream-detection}节说明流探测技术,该算法是Rat编译实现中最为关键的算法,是并行任务生成的基础。
%% 第\ref{sec:np-decomposition}节说明嵌套并行的分解技术,它的功能是将嵌套并行操作分解
%% 为一维并行操作序列。
第\ref{sec:stream-optimization}节介绍若干高层优化技术,这些高层优化技术
施加在向量流上。
第\ref{sec:code-generation}节介绍如何从程序的向量流图生成C代码。

\section{编译实现总体方案}\label{sec:compiler-overview}
第\ref{sec:c-interface}节已经提到,Rat编译器是一个源到源编译期,
最终产生C语言代码,将Rat函数转换成合适类型的C语言函数供C程序调用。
图\ref{fig:frontend}给出了从Rat代码到C代码的翻译流程,步骤如下:
\begin{enumerate}
  \item Rat代码经过词法分析器与语法分析器,形成L1语法树。
    L1语法树是Rat代码的直接表示,它包含的结点类型参见表\ref{tbl:l1-ast}。
  \item L1语法树经过L1变换得到L2语法树。L2语法树结构更加简单,包含的结点类型
    更少,参见表\ref{tbl:l2-ast}。
  \item L2语法树经过L2变换得到Core语法树。Core语法树是Core语言的树状表示,
    Core语法树的结点类型同L2语法树。
  \item Core语法树经过流探测算法转化成向量流图表示。
  \item 在向量流图层面实施若干高层优化技术。
  \item 程序从向量流图形式最后被翻译成C语言程序,该C程序是对Rat运行时系统的API调用,
    运行时系统将在下一章介绍。
\end{enumerate}
\begin{figure}[tbh]
  \centering
  \includegraphics[width=\linewidth]{frontend}
  \caption{Rat编译流程}
  \label{fig:frontend}
\end{figure}
\begin{table}[tbh]
  \centering
  \caption{L1语法树结点类型}\label{tbl:l1-ast}
  \begin{tabularx}{\linewidth}{ZZ}
      \toprule[1.5pt]
      {\hei 结点类型} & {\hei 描述} \\
      \midrule[1pt]
      type-define & 类型定义\\
      object-decl & 对象声明\\
      object-def & 对象定义\\
      object-ref & 对象引用\\
      literal & 字面量\\
      function-app & 函数应用\\
      lambda-exp & 函数定义\\
      conditional & 条件表达式\\
      vector-comprehension & 向量推导\\
      vector-ele-ref & 向量元素引用\\
      vector-slice-ref & 向量片段引用\\
      local-binding & 局部定义\\
      \bottomrule[1.5pt]
    \end{tabularx}
\end{table}
\begin{table}[tbh]
  \centering
  \caption{L2语法树与Core语法树结点类型}\label{tbl:l2-ast}
  \begin{tabularx}{\linewidth}{ZZ}
      \toprule[1.5pt]
      {\hei 结点类型} & {\hei 描述} \\
      \midrule[1pt]
      object-ref & 对象引用\\
      literal & 字面量\\
      function-app & 函数应用\\
      lambda-exp & 函数定义\\
      conditional & 条件表达式\\
      \bottomrule[1.5pt]
    \end{tabularx}
\end{table}

\section{Core语言}\label{sec:core-language}
Core语言是一种结构简单的语言,容易分析,并且支持若干高层优化技术。

Core语言的指令集庞大,但结构相对简单,只包含三类指令:向量原语、标量原语和
辅助指令。
其中,标量原语提供了数值计算能力,
向量原语提供了并行计算能力。两类指令构成了一种简单而强大
的计算模型,非常适合表达实际的数据并行问题。
辅助指令提供一些附加功能。

\subsection{向量原语}\label{subsec:vector-primitives}
Core语言提供了一组非常精简的并行操作原语,它包括了一些最为基本的向量操作,这些操作
最为重要的特点是非常适合并行实现。
事实上,正是这组原语为Rat提供了数据并行能力。
表\ref{tbl:vector-primitives}列出了所有向量原语。
\begin{table}[htb]
  \centering
  \caption{向量原语}
  \label{tbl:vector-primitives}
  \begin{tabularx}{\linewidth}{p{10em}X}
    \toprule[1.5pt]
    \hei{向量原语} & \hei{功能说明} \\
    \midrule[1pt]
    \texttt{map} & 给定一元操作,对一个向量中所有元素施加同一操作\\
    \texttt{scan} & 给定满足结合律的二元操作,求一个向量中元素的部分和\\
    \texttt{gpermute} & 对向量中的元素重排\\
    \texttt{gcopy} & 对向量中的元素采取特定步长复制\\
    \texttt{shrink} & 对给定向量采取紧缩复制\\
    \texttt{sort} & 给定排序函数,对向量中的元素进行排序\\
    \texttt{random} & 产生随机向量\\
    \texttt{generate} & 根据位置产生向量元素\\
    \texttt{zip} & 将多个向量中的元素逐位置组成元组\\
    \bottomrule[1.5pt]
  \end{tabularx}
\end{table}

这组原语存在于典型的函数式语言中,但一般是针对表(List)定义的,其语义也是串行的,
但Core语言的向量原语是并行的,这是一个显著区别。
这组原语的选取受到了MapReduce的启发\upcite{Dean2008}
使用这组精简的原语集可以构建更多更为复杂的向量操作,
如表\ref{tbl:vector-operations}中列出的\texttt{filter}操作就可以按照如下方式定义:\\
\centerline{\texttt{filter p v = shrink (scan (map p v)) v}}

下面分别给出各向量原语的定义。
\subsubsection{map}
\begin{definition}
  map原语以一个一元函数$f$与一个向量$$[a_0, a_1, \cdots, a_{n-1}]$$作为输入参数,
  返回一个向量$$[f(a_0), f(a_1), \cdots, f(a_{n-1})]$$作为结果。
\end{definition}

图\ref{fig:map-diagram}给出map原语的操作说明。
\begin{figure}[h]
  \centering
  \includegraphics[height=4cm]{map}
  \caption{map原语操作示意图}
  \label{fig:map-diagram}
\end{figure}

\subsubsection{scan}
\begin{definition}
  scan原语以一个满足结合律以$I$为幺元的二元函数$\oplus$与一个向量$$[a_0, a_1, \cdots, a_{n-1}]$$作为输入,
  根据两个配置参数  $ScanDirection, InclusiveMode$的取值返回一个部分和向量作为结果。
  不同配置参数下的返回结果如下:\\
  $[a_0, (a_0\oplus{}a_1), \cdots, (a_0\oplus{}\cdots\oplus{}a_{n-1})]$\hfill{}Inclusive, Forward\\
  $[(a_0\oplus\cdots\oplus{}a_{n-1}), \cdots, (a_{n-2}\oplus{}a_{n-1}), a_{n-1}]$\hfill{}Inclusive, Backward\\
  $[I, a_0, (a_0\oplus{}a_1), \cdots, (a_0\oplus{}\cdots\oplus{}a_{n-2})]$\hfill{}Exclusive, Forward\\
  $[(a_1\oplus\cdots\oplus{}a_{n-1}), \cdots, (a_{n-2}\oplus{}a_{n-1}), a_{n-1}, I]$\hfill{}Exclusive, Backward\\
\end{definition}

图\ref{fig:scan-diagram}给出scan原语的操作说明。
\begin{figure}[h]
  \centering
  \subfloat[Inclusive, Forward]{
    \includegraphics[height=4cm]{scan-if}\hspace{2em}
  }
  \subfloat[Exclusive, Forward]{
    \includegraphics[height=4cm]{scan-ef}\hspace{2em}
  }
  \\
  \subfloat[Inclusive, Backward]{
    \includegraphics[height=4cm]{scan-ib}\hspace{2em}
  }
  \subfloat[Exclusive, Backward]{
    \includegraphics[height=4cm]{scan-eb}\hspace{2em}
  }
  \caption{scan原语操作示意图}
  \label{fig:scan-diagram}
\end{figure}

\subsubsection{gpermute}
\begin{definition}
  gpermute原语以一个索引计算函数$g$与一个向量$$[a_0, a_1, \cdots, a_{n-1}]$$作为输入,
  根据配置参数  $PermuteDirection$的取值返回一个重排序的向量作为结果。
  不同配置参数下的返回结果如下:\\
  $[a_{g^{-1}(0)}, a_{g^{-1}(1)}, \cdots, a_{g^{-1}_{n-1}}]$\hfill{}Forward\\
  $[a_{g(0)}, a_{g(1)}, \cdots, a_{g_{n-1}}]$\hfill{}Backward\\  
\end{definition}

图\ref{fig:gpermute-diagram}给出gpermute原语的操作说明。
\begin{figure}[h]
  \centering
  \subfloat[Forward]{
    \includegraphics[height=4cm]{gpermute-f}
  }
  \\
  \subfloat[Backward]{
    \includegraphics[height=4cm]{gpermute-b}
  }
  \caption{gpermute原语操作示意图}
  \label{fig:gpermute-diagram}
\end{figure}

\subsubsection{gcopy}
\begin{definition}
  gcopy原语以一个整数三元组$(i, j, s)$与一个向量$$[a_0, a_1, \cdots, a_{n-1}]$$作为输入参数,
  其中$0\le{}i<j\le{}n-1$,返回一个向量$$[a_{i}, a_{i+s}, a_{i+2s}, \cdots, a_{j-mod(j-i,s)}]$$作为结果。
\end{definition}

图\ref{fig:gcopy-diagram}给出gcopy原语的操作说明。
\begin{figure}[h]
  \centering
  \includegraphics[height=4cm]{gcopy}
  \caption{gcopy原语操作示意图}
  \label{fig:gcopy-diagram}
\end{figure}

\subsubsection{shrink}
\begin{definition}
  shrink原语以一个位置压缩向量$$[b_0, b_1, \cdots, b_{n-1}]$$与一个数据向量$$[a_0, a_1, \cdots, a_{n-1}]$$作为输入,
  返回一个向量$$[a_{n_1}, a_{n_2}, \cdots, a_{n_k}]$$作为结果。
\end{definition}

图\ref{fig:shrink-diagram}给出shrink原语的操作说明。
\begin{figure}[h]
  \centering
  \includegraphics[height=4cm]{shrink}
  \caption{shrink原语操作示意图}
  \label{fig:shrink-diagram}
\end{figure}

\subsubsection{sort}
\begin{definition}
  sort原语以一个二元比较函数$\prec$与一个向量$$[a_0, a_1, \cdots, a_{n-1}]$$作为输入,
  返回一个结果向量$$[a_{n_1}, a_{n_2}, \cdots, a_{n_n}]$$,该向量是输入向量的一个重排,
  并且对于任意$n_i\prec{}n_j$都有$a{n_i}\prec{}a{n_j}$。
\end{definition}

图\ref{fig:sort-diagram}给出sort原语的操作说明。
\begin{figure}[h]
  \centering
  \includegraphics[height=4cm]{sort}
  \caption{sort原语操作示意图}
  \label{fig:sort-diagram}
\end{figure}

\subsubsection{zip}
图\ref{fig:zip-diagram}给出了zip原语的操作说明。
\begin{definition}
  zip原语以两个等长向量$$[a_0, a_1, \cdots, a_{n-1}]$$ $$[b_0, b_1, \cdots, b_{n-1}]$$作为输入参数,
  返回一个向量$$[(a_0, b_0), (a_1, b_1), \cdots, (a_{n-1}, b_{n-1})]$$作为结果。
\end{definition}

图\ref{fig:zip-diagram}给出zip原语的操作说明。
\begin{figure}[h]
  \centering
  \includegraphics[height=4cm]{zip}
  \caption{zip原语操作示意图}
  \label{fig:zip-diagram}
\end{figure}

\subsection{标量原语}\label{subsec:scalar-primitives}
Core语言提供了数量众多的标量原语,功能为基本的算数逻辑运算与常用数学函数,
这其中囊括了标准C数学库定义的所有数学函数,附录\ref{chap:core-scalar-primitives}给出了
Core语言标量原语的完整列表。

标量原语中有一类需要注意,即某些满足结合律同时存在幺元的操作,
如加法运算、乘法运算、布尔数求与、布尔数求或等。
从代数学的角度上讲,这种运算可以与某种数据类型构成一个幺半群。

对于向量原语\texttt{scan},只有当传递给它的二元操作满足上述性质时,
\texttt{scan}才可以并行执行,所以,只有满足该性质的标量原语可以作为
参数传递给\texttt{scan}。表\ref{tbl:monoid-scalar-primitives}列出了所有满足这种性质
的标量原语。
\begin{table}
  \centering
  \caption{满足结合律的标量原语}
  \label{tbl:monoid-scalar-primitives}
  \begin{tabularx}{\linewidth}{ccXX}
    \toprule[1.5pt]
    \hei{标量原语} & \hei{功能说明} & \hei{幺半群类型} & \hei{幺元}\\
    \midrule[1pt]
    \texttt{+} & 加法 & 所有数值类型 & 0\\
    \texttt{*} & 乘法 & 所有数值类型 & 1\\
    \texttt{and} & 逻辑与 & 整数类型 & \texttt{true}\\
    \texttt{or} & 逻辑或 & 整数类型 & \texttt{false}\\
    \texttt{max} & 最大值 & 所有数值类型 & 该类型可表示的极大值\\
    \texttt{min} & 最小值 & 所有数值类型 & 该类型可表示的极小值\\
    %% \texttt{gcd} & 最大公约数 & 所有整数类型 & \\
    %% \texttt{lcm} & 最大公约数 & 所有整数类型 & \\
    \bottomrule[1.5pt]
  \end{tabularx}
\end{table}

\subsection{辅助指令}
Core语言提供了若干的辅助指令,这部分指令既不属于提供数值计算能力的
标量原语,也不属于提供并行计算能力的向量原语。
表\ref{tbl:assist-instruction}列出了他们的名称及功能。
\begin{table}
  \centering
  \caption{辅助指令}
  \label{tbl:assist-instruction}
  \begin{tabularx}{\linewidth}{p{10em}X}
    \toprule[1.5pt]
    \hei{辅助指令} & \hei{功能说明} \\
    \midrule[1pt]
    \texttt{length} & 返回向量长度\\
    \texttt{vref} & 引用向量某个特定位置的值\\
    %% \texttt{collect} & 将多个同类型数据收集成为一个向量\\
    \texttt{if} & 条件指令,根据条件表达式的值返回不同的值\\
    \texttt{foreach} & 迭代指令,将多个同类型数据收集成为一个向量\\
    \bottomrule[1.5pt]
  \end{tabularx}
\end{table}

需要注意的是,表中的\texttt{if}指令与\texttt{foreach}指令不同于命令式语言
中的相应概念,在命令式语言中二者是一种语法结构,而在Core语言中这两条都是指令,
它们与操作数共同构成表达式,代表某个特定的值。
以\texttt{foreach}指令为例,它以一个向量\texttt{v}
与一个操作\texttt{op}为参数,将\texttt{op}作用在\texttt{v}中所有
元素之上,然后将结果再次收集成一个向量。可以将\texttt{foreach}看作
一种和\texttt{map}类似的操作,他们区别在于\texttt{map}是并行执行的,
而\texttt{foreach}指令则不一定如此。
\texttt{foreach}指令与嵌套循环的处理有关,这将在第\ref{subsec:np-decomposition}节说明。

\section{流探测技术}\label{sec:stream-detection}
在流驱动计算模型中,问题被表示为数据流图的形式,
数据之间存在的依赖关系由数据流图显式给出,计算过程
沿着数据流的方向前进。一旦某个计算任务的所有输入数据
变成可用状态,那么该计算任务就可以执行,并产生新的数据输出。

Rat程序的并行执行采用流驱动计算模型\upcite{Johnston2004},在研究现状中提及流驱动并行语言。
流探测技术是Rat程序并行化的关键技术,是程序自动并行化的基础。
第\ref{subsec:stream-concept}节将首先结合实例说明流的概念,
第\ref{subsec:stream-detection-algorithm}给出流探测算法的正规描述。
%% 第\ref{subsec:task-generation}最后介绍根据流探测算法的结果静态产生
%% 计算任务的方法。

%% 流探测结果生成流探测算法根据程序的Core语法树构建出向量流图。
%% 进而从向量流图中可以搜索可并行任务,
%% 同时在单一流上可以实施上一节中提出的高层优化技术。

\subsection{向量流图}\label{subsec:stream-concept}
第\ref{subsec:functional-advantages}节中提到,纯函数特性使得表达式的求值结果与子表达式求值顺序无关,
用树的形式表示,就是整个表达式树的求值结果与各个子树的求值顺序无关。用向量流图表示,
就是不同流的计算过程可以独立执行,他们的执行顺序不影响最终结果。
同时,将流的概念抽象出来,也为应用第\ref{sec:stream-optimization}节提出的流优化技术
创造了条件。

我们将连续的向量操作序列与这些操作输出的向量序列称之为“流”,其定义如下:
\begin{definition}
  称施加在某个数据向量上的向量操作序列为程序中任务流,由这个向量操作序列生成的向量数据序列
  称为程序中的向量流。
\end{definition}

在给出流探测算法的正规描述之前,先结合并行问题实例进行说明。
\begin{quotation}
  \kai{
    以第\ref{sec:n-body}节中的n-body问题为例,该问题的Core语法树见图\ref{fig:n-body-core},
    (为清晰起见,该图进行了一定的简化,原始的Core语法树规模更为庞大)。
    从图中可以看出,输入向量\texttt{Cells}(图中用矩形表示)一共被引用了五次,
    从\texttt{Cells}出发,经过一些向量原语的操作,生成了若干个中间变量(图中用椭圆形表示),
    如\texttt{positions}与\texttt{velocities}等,这些中间变量再通过其他向量操作最后生成了
    输出结果。

    通过流探测算法生成的向量流如图\ref{fig:n-body-stream}。该图清晰地表示出数据的流动过程,
    图中每一个结点表示一个向量,每一条边代表某种向量操作。
  }
\end{quotation}
\begin{figure}
  \centering
  \includegraphics[width=\linewidth]{n-body-core-new}
  \caption{n-body问题Core语法树}
  \label{fig:n-body-core}
\end{figure}
\begin{figure}
  \centering
  \includegraphics[width=\linewidth]{n-body-stream-new}
  \caption{n-body问题向量流图}
  \label{fig:n-body-stream}
\end{figure}

向量流图中每一条有向路径代表了一条向量流。当某个向量结点\texttt{v}有多于一条入边时,称这些边所在的
向量流在\texttt{v}处\emph{汇合},当某个向量结点u有多于一条出边时,称输入\texttt{u}
的向量流在\texttt{u}处\emph{分支},汇合操作是由\texttt{zip}原语引起的,分支操作是因为
相应的向量被多个向量操作引用。

向量流图的对偶图即为任务流图。此处约定,后文中主要以向量流为分析对象,
如无特别说明,文中提及的“流”即为向量流。

对比观察\ref{fig:n-body-core}与
图\ref{fig:n-body-stream}可以发现,表达式树与向量流图都可以反映整个表达式的求值过程,只不过表达式树采用
自顶向下的求值逻辑,其含义是:如果想要求解某个表达式的值,那么要先求解它依赖的子表达式的值。
而向量流图采用了流驱动(data-flow driven)方式\upcite{Johnston2004},
其含义是:一旦某条流中的某个结点值可用,那么从该结点
分支出去的所有流都可以继续向前计算。

%% 综上,将程序中的流抽象出来有两方面作用:
%% \begin{compactitem}
%%   \item 不同流的计算可以独立并行执行;
%%   \item 在流上可以实施向量原语重排、向量原语聚合三种优化技术。
%% \end{compactitem}

\subsection{流探测算法}\label{subsec:stream-detection-algorithm}
算法\ref{alg:stream-detection}给出了流探测算法的正规描述。

流探测算法是一个以
树的深度优先遍历\upcite{Tarjan1972}为基础的递归算法,根据Core语法树不同结点类型做不同处理,
流探测算法处理的结点类型包括对象引用结点(object-ref)、
函数应用结点(function-app)、$\lambda$表达式结点(lambda-exp)、
条件分支结点(conditional),字面量结点(literal)无需处理。
Core语法树结点类型参见表\ref{tbl:l2-ast},
\begin{compactitem}
  \item 第2-3行,如果当前结点是对象引用结点,且该对象的表达式还未被探测过,则
    对该对象调用递归探测算法。
  \item 第4-23行,如果当前结点是函数应用结点
    \begin{compactitem}
      \item 第6-18行,如果当前结点调用的函数是向量原语
        \begin{compactitem}
          \item 第7-11行,对该操作施加的每一个输入向量建立一条边,
            从输入向量指向当前结点,并设置该边的相关属性。
          \item 第12-18行,若为\texttt{map}操作,那么对\texttt{map}施加的标量探测
            调用递归探测算法。如果该标量操作内部包含了向量操作,则设置本节点的边属性为嵌套并行。
        \end{compactitem}
      \item 第19-21行,如果当前结点调用的函数是标量操作,则对该操作调用递归探测算法。
    \end{compactitem}
  \item 第22-23行,如果当前结点是$\lambda$表达式结点,则对它的返回值表达式调用递归探测算法。
  \item 第24-32行,如果当前结点是条件分支结点,则对每一个结果为向量类型可能的分支建立一条边,
    并标记这条边的属性为条件分支,该边是否有效将在运行时确定。
\end{compactitem}
\begin{algorithm}[htbp]
  \caption{流探测算法}
  \label{alg:stream-detection}
  \begin{algorithmic}[1]
    \Require 待求值表达式Core语法树结点$T$,已经建立的向量流图$V$
    \Ensure 处理完当前结点$T$之后的向量流图$V$
    %% \Function{detect-stream}{$T$}
    %%   \State $V \leftarrow \left\{ T \right\}$
    %%   \State \Return \Call{detect-stream-rec}{$T, V$}
    %% \EndFunction
    \Function{detect-stream-rec}{$n, V$}
      \If{$nodetype(n) = OBJECT\_REF \ \&\  is\_detected(n) = FALSE$}
      \State $V \leftarrow$ \Call{detect-stream-rec}{$ref(n), V$}
      %% LITERAL
      %% \ElsIf{$nodetype(n) = LITERAL$}
      %% \State $donothing$
      %% FUNCTION_APP
      \ElsIf{$nodetype(n) = FUNCTION\_APP$}
      \State $operator \leftarrow get\_operator(T)$ % get operator
      \If{$is\_vector\_primitive(operator)$}        % vector primitive
      \For{$operand$ in $get\_operands(n) \& is\_vector(operand)$}
      \State $operand.add\_child(n)$
      \State $operand.set\_op\_(n, operator)$
      \State $V \leftarrow V \cup \left\{ operand \right\}$
      \EndFor
      \If{$operator = MAP$}                         % map
      \State $sop \leftarrow get\_scalar\_op(T)$
      \State $V \leftarrow$ \Call{detect-stream-rec}{$sop, V$}
      \If{$contain\_vop(sop)$}
      \State $operand.set\_flag(NESTED\_PARALLEL)$
      \EndIf
      \EndIf
      \Else
      \State $V \leftarrow$ \Call{detect-stream-rec}{$operator, V$}
      \EndIf
      \ElsIf{$nodetype(n) = LAMBDA\_EXP$}
      \State $V \leftarrow$ \Call{detect-stream-rec}{$get\_value(n)$}
      \ElsIf{$nodetype(n) = CONDITIONAL$}
      \For{$branch$ in $get\_branches(n)$} {
        \If{$is\_vector(branch)$}
        \State $branch.add\_child(n)$
        \State $branch.set\_flag(n, CONDITIONAL)$
        \State $V \leftarrow V \cup \left\{ branch \right\}$
        \EndIf
      }
      \EndFor
      \EndIf
      \State \Return $V$
    \EndFunction
  \end{algorithmic}
\end{algorithm}

算法\ref{alg:stream-detection}中只给出了流探测算法的主干,
该算法在探测流的同时还完成一些标记工作,包括所有数据结点与变量结点的引用计数统计与依赖计数统计等。
这些静态执行的标记工作将被Rat的运行时系统用来完成内存管理等任务,


\section{流优化技术}\label{sec:stream-optimization}
执行流探测算法之后,Core表达式求值树被转化为向量流图,
本节提出两种施加在流上的优化技术:向量原语重排与向量原语聚合,
下面分别介绍两种优化技术。
%% ,然后给出流优化算法。

\subsection{向量原语重排}
向量原语重排(reorder)是指,在某些情况下,相邻的向量操作可以通过交换顺序
减少需要处理的数据总量,从而达到降低线程资源、存储器空间、访存次数等方面的开销,
下面举例说明向量原语重排的应用。
\begin{quotation}
  \kai{
    考虑下面的表达式求值问题:\\
    \centerline{\texttt{gcopy (m, n, 1) (map f input)}}\\
    该表达式等价于:\\
    \centerline{\texttt{map f (gcopy (m, n, 1) input)}}\\
    两个表达式语义上等价,但求解过程不同,参见图\ref{fig:vp-reorder}。从图\ref{fig:vp-reorder}
    可以看出,第二个表达式要比第少个表达式多处理一部分数据,这样就能减少需要的线程资源,节省
    存储器带宽,即通过将\texttt{map}原语与\texttt{slice}“重排”执行能够提高效率。
  }
\end{quotation}
\begin{figure}
  \centering
  \subfloat[\texttt{gcopy (m, n, 1, 1) (map f input)}]{
    \includegraphics[height=4cm]{vp-reorder-1}
  }
  \\
  \subfloat[\texttt{map f (gcopy (m, n, 1, 1) input)}]{
    \includegraphics[height=4cm]{vp-reorder-2}
  }
  \caption{向量原语重排示例}
  \label{fig:vp-reorder}
\end{figure}

在表\ref{tbl:vector-primitives}列出的向量原语中,\texttt{shrink}与\texttt{gcopy}原语的输出向量
包含的元素都是输入向量的一个子集。如果一个\texttt{map}操作之后紧跟着
一个\texttt{shrink}操作或\texttt{gcopy}操作,那么\texttt{map}完成的部分工作就会被舍弃掉。
这时,可以交换\texttt{map}与\texttt{shrink}或\texttt{gcopy}的操作顺序,就可以减少\texttt{map}
需要处理的数据量,提高效率。

%% 向量原语两两之间的可重排性参见表\ref{tbl:vp-reorder},
%% 其中每个单元格对应的两个向量原语的执行顺序为“先左边,再上方”。
%% 标注$\surd$的单元格表示,该格对应的两个向量原语应该重排以提高执行效率;
%% 标注--的单元格表示该格对应的两个向量原语可以重排,但不会带来效率提升;
%% 标注$\times$的单元格表示该格对应的两个向量原语不应重排,重排会带来性能损失;
%% 空单元格表示该格对应的两个向量原语不能重排,重排会造成语义错误。
%% \begin{table}
%%   \centering
%%   \caption{向量原语可重排性}\label{tbl:vp-reorder}
%%   \begin{tabularx}{\linewidth}{|c|Z|Z|Z|Z|Z|c|Z|Z|}
%%     \hline
%%     & \texttt{map} & \texttt{slice} & \texttt{concat} & \texttt{zip} &
%%     \texttt{scan} & \texttt{gpermute} & \texttt{gcopy} & \texttt{sort}\\
%%     \hline
%%     \texttt{map} & -- & $\surd$ & & & & -- & $\surd$ & \\
%%     \hline
%%     \texttt{slice} & $\times$ & & & & & & & \\
%%     \hline
%%     \texttt{concat} & & & & & & & &\\
%%     \hline
%%     \texttt{zip} & & & & & & & &\\
%%     \hline
%%     \texttt{scan} & & & & & & & &\\
%%     \hline
%%     \texttt{gpermute} & -- & & & & & & & \\
%%     \hline
%%     \texttt{gcopy} & $\times$ & & & & & & & \\
%%     \hline
%%     \texttt{sort} & & & & & & & &\\
%%     \hline
%%   \end{tabularx}
%% \end{table}

\subsection{向量原语聚合}
向量原语聚合(fusion)是指,在某些情况下,多个向量操作可以合并成单个向量操作,
从而可以减少访存,节省存储器带宽,同时还能消除访问存储器造成的时延,提高执行速度。
下面举例说明向量原语聚合的应用。
\begin{quotation}
  \kai{
    考虑下面的表达式求值问题:\\
    \centerline{\texttt{map f (map g input)}}\\
    该表达式等价于:\\
    \centerline{\texttt{map (f . g) input}}\\
    两个表达式语义上等价,但求解过程不同,参见图\ref{fig:vp-fusion}。从图\ref{fig:vp-fusion}
    可以看出,第一个表达式比第二个表达式多引入一个中间向量,虽然
    通过前一小节介绍的存储空间优化技术避免为这个中间向量分配新的的存储空间,但仍然会多引入
    一次的向量读与一次向量写。此时,将两个\texttt{map}原语“聚合”成一次能够提高效率。
  }
\end{quotation}
\begin{figure}
  \centering
  \subfloat[\texttt{map f (map g input)}]{
    \includegraphics[height=4cm]{vp-fusion-1}
  }
  \\
  \subfloat[\texttt{map (f . g) input}]{
    \includegraphics[height=4cm]{vp-fusion-2}
  }
  \caption{向量原语聚合示例}
  \label{fig:vp-fusion}
\end{figure}

表\ref{tbl:vp-fusion}列出了所有两两之间存在聚合能力的向量原语对。
\begin{table}
  \centering
  \caption{可聚合向量原语}\label{tbl:vp-fusion}
  \begin{tabularx}{\linewidth}{XX}
    \toprule[1.5pt]
    \hei{聚合前件} & \hei{聚合后件}\\
    \midrule[1pt]
    \texttt{map} & \texttt{map, gpermute, gcopy}\\
    \texttt{gpermute} & \texttt{map, gpermute, gcopy, sort}\\
    \texttt{gcopy} & \texttt{map, gpermute, gcopy}\\
    \texttt{shrink} & \texttt{map}\\
    \bottomrule[1.5pt]
  \end{tabularx}
\end{table}
%% \subsection{流优化算法}
%% 前面介绍了两种流优化技术,这里给出流优化算法的正规描述。

%% \section{并行虚拟机设计}\label{sec:parallel-vm}
%% %% Rat的并行计算能力由向量原语提供,向量原语在多核处理器与GPU上都可以高效实现。
%% Rat实现方案采用了层次化软件设计,并行虚拟机CoreVM就是这个层次化设计的关键。
%% CoreVM是一个抽象层次较高的虚拟机,它有下面两个特点:
%% \begin{compactitem}
%%   \item CoreVM独立于具体的并行硬件结构设计;
%%   \item CoreVM在保持高抽象层次的同时也能够很好地描述并行硬件,
%%     原理上可以在多核处理器以及GPU等不同并行硬件上高效实现。
%% \end{compactitem}
%% 这样,只需要在不同的并行硬件上实现CoreVM,就可以完成Rat程序
%% 在不同硬件上的移植。

%% 第\ref{subsec:core-language}先介绍Core语言,它是并行虚拟机CoreVM运行的指令集,
%% 然后第\ref{subsec:stream-optimization}节介绍几种执行在Core语言层面执行的优化技术,
%% %% 最后第\ref{subsec:}给出



%% \section{自动并行化技术}\label{sec:auto-parallelization}
%% %% 正如第\ref{sec:parallel-vm}节所说,Rat的并行虚拟机CoreVM被设计为在多种
%% %% 并行硬件上都能够高效实现,这实现了本章开始提出的三个目标中的第一个。
%% %% Rat的第一版实现方案选择Nvidia公司的GPU作为运行CoreVM的并行硬件,
%% %% 在实现任务自动并行化的时候着力与完成另外两个目标,即充分发掘GPU的并行计算能力与
%% %% 有效协调CPU与GPU并行工作。
%% 任务的自动并行化一直是一个难题,因为并行问题的固有特征导致不同问题可发掘的并行潜力各不相同,
%% 这使得对与并行问题的求解没有通用的最优化方案。CoreVM的设计使得数据并行问题被分解成
%% 一个个向量操作组成的序列,而这个将并行问题序列化为向量操作的过程,
%% 消除了问题本身的固有特性,这时,序列化的向量操作成为一种“通用”的并行问题描述形式,
%% 这为发现问题无关的并行性提供了可能。

%% 本节介绍Core程序自动并行化方案,内容安排如下:
%% 第\ref{subsec:stream-detection}节将首先介绍
%% 流探测(stream detection)算法,该算法将并行问题转化成流的描述形式,
%% 为发现问题中的可并行任务做前期准备工作。
%% 第\ref{subsec:np-decomposition}节说明嵌套并行的分解(decomposition)技术,
%% 这种技术用于处理嵌套并行操作,将嵌套并行转化为一维向量操作。



%% %% \subsection{嵌套并行平坦化}\label{subsec:np-flattening}
%% %% 嵌套并行的分解算法思路直观,实现简单,但是有可能会造成一些性能上的问题。
%% %% 当嵌套循环的最内层向量操作规模不够大的时候,会造成大批量的细粒度任务,
%% %% 这不仅有可能造成并行硬件的计算能力得不到充分利用,还可能会显著提高
%% %% 任务调度器的调度开销在整个程序中所占的比例。

%% %% 为解决上述问题,这里提出了嵌套并行平坦化技术作为嵌套并行分解的辅助措施。
%% %% 图\ref{fig:np-decomposition}中表示的一组流结构相同,仅存在一些参数上的差异。
%% %% 可以看出,结构相同是因为嵌套并行中对每一个外层向量数据,内层并行操作都
%% %% 保持相同的计算逻辑,参数不同是因为每一次内层向量操作引用了外层向量不同位置的元素,
%% %% 所以,区别这些流的参数是一个整型数,代表外层向量中被引用元素的位置。

\section{代码生成}\label{sec:code-generation}
在向量流图上实施流优化算法以后,Rat编译期将把优化过的向量流图翻译成C程序,
该C程序是一个对Rat运行时系统API的调用序列。C程序中,每一个向量操作使用一个
任务结构表示,程序中所有任务都在这一阶段由编译期静态生成,然后产生一个任务
初始化函数将所有任务添加到运行时系统的任务调度队列,在程序启动后,运行时
系统的任务调度器会负责调度任务执行。Rat编译器生成的C程序框架如下:
\lstinputlisting[language=C]{listings/skeleton.c}

第\ref{subsec:task-encapsulation}节介绍如何将向量流图中的向量操作映射到C语言任务结构,
第\ref{subsec:np-decomposition}节介绍嵌套并行分解技术,将嵌套并行操作
转化成一维并行操作序列。

\subsection{任务封装}\label{subsec:task-encapsulation}
任务(task)是Rat运行时系统执行计算的粒度单元,每一个任务都封装了一个向量原语操作,
它由下列三个要素构成:输入向量、输出向量与操作类型。
在向量流图中,每一条有向边(或一组汇入同一结点的有向边)都表示一个向量原语操作,
边的起点表示该操作的输入向量,终点表示该操作的输出向量。

C语言中的任务结构\texttt{task}定义在下面的代码中给出,
该结构中的\texttt{vp}成员指向具体的向量原语结构。
\texttt{in}与\texttt{out}成员指向输入与输出对象,
\texttt{vector}也是结构体类型,他们内部有指向存放向量数据
的内存区域的指针,但在编译期不为向量对象分配内存,
向量内存是由运行时系统动态管理的。

不同的向量原语有不同结构定义,如\texttt{map}原语的结构类型包括了一个函数指针\texttt{op},
指向要施加在向量上的标量操作,\texttt{intype}与\texttt{output}确定
该操作的实际类型。
\lstinputlisting[language=C]{listings/task.c}

向量流图中的每一条边都被打包成一个\texttt{task}结构,
然后通过运行时系统API装入任务调度器。这种封装方式保存了向量流图
的逻辑结构,运行时系统将继续依靠向量流图实现任务的动态调度。

因为从同一结点出发的各条边代表的向量操作可以并行执行,所以编译期任务生成
采用了一种简单策略,即对向量流图中的任务采用宽度优先序排列:
从程序的输入向量为出发结点集合,对向量流图执行一次宽度优先遍历,
每发现一条边,就将改边代表的向量操作封装成
一个任务,提交给任务调度器。

任务的宽度优先序只是静态产生的“推荐”调度顺序,在这个任务排序中,相邻的任务
能够并行执行的“可能性”较高。但在程序的运行期,所有添加到调度器的任务是被
动态调度的,这将在下一章中详细说明。

有一条指令需要特殊处理,那就是\texttt{foreach}指令。
该指令生成的不是一个单一任务,而是一个任务队列,
在C代码中表现为一次循环,循环中每次迭代都执行一些任务装载工作(参见本节开始
列出的代码片段)。
这是由嵌套并行造成的,下面的章节将对这种情况进行说明。

\subsection{嵌套并行分解}\label{subsec:np-decomposition}
前文节中已经提及,虽然Rat只提供了一维向量原语,但由于支持高阶函数,
编程者可以使用一维向量原语表达高维的并行操作,即支持嵌套并行。

简而言之,嵌套并行是因为在一个标量操作的内部逻辑中引入了向量操作。
一维并行与嵌套并行的关系类似于串行语言中单层循环与嵌套循环的关系,
如果将一维并行操作看作单层循环的并行版本,那么嵌套并行就是嵌套循环
的并行版本。
\begin{quotation}
  \kai{
    在\texttt{n-body}问题的向量流图\ref{fig:n-body-stream}中,有一条用虚线绘制的有向边,
    该边从\texttt{cells}结点指向\texttt{accelerates}结点。
    \texttt{accelerates}结点的求值表达式为\texttt{map (calcSingleAcc cell) cells},
    因为\texttt{calcAccelerate}并不是一个向量数据,而是一个函数,
    是一个由\texttt{map}施加在\texttt{cells}之上的标量操作。
    它之所以出现在向量流图中,是因为它的内部引用了两个向量操作\texttt{map}
    与\texttt{fold}。也就是说,虽然从类型声明上看,\texttt{calcAccelerate}的输入输出
    都是标量,但它并不是一个“真正”的标量操作。图\ref{fig:n-body-stream}中采用
    虚线标识出这两个结点的关系正是因为这条边并不能构成一条真正的流。
  }
\end{quotation}

嵌套并行分解就是将嵌套并行操作分解为多个一维并行操作,这样,
整个任务就可以完全使用一维并行原语实现。

回顾表\ref{tbl:vector-primitives}列出
的向量原语,其中\texttt{map},\texttt{scan},\texttt{gpermute},\texttt{sort}
在参数中引入了标量操作,从而可能引起嵌套并行的出现,而在实践当中,
一般只有\texttt{map}原语引用的标量操作会在内部会包含向量操作,造成嵌套并行,
因而,下面我们只分解考虑\texttt{map}原语引入嵌套并行。

%% 在将程序最终翻译成C程序的时候,向量流图中每一个向量操作都会被封装成一个任务结构发送给任务调度器,
%% 具体形式就是一个运行时系统API调用。
对于\texttt{map}原语引入的嵌套并行操作,最简单的策略是,将这个\texttt{map}
操作翻译成一个循环体,其迭代次数为\texttt{map}原语目标向量的长度,循环体内部
为目标向量的每一个元素都生成一个任务发送到任务调度器。这种处理策略的好处是
思路简单,而且支持任意多层嵌套并行的处理,嵌套并行最终被转化成一批等规模的一维向量操作,
规模等于最内层向量操作的规模。

一般地,对于一个嵌套并行操作\texttt{map sopContainingVop v}
我们将它转化成下面的形式:\texttt{foreach v sopContainingVop}
正如前面一节所说,\texttt{foreach}指令将生成一个任务队列。

%% 算法\ref{alg:np-decomposition}给出了
%% 这种简单嵌套并行分解算法的正规描述。
%% \begin{algorithm}
%%   \caption{简单的嵌套并行分解算法}
%%   \label{alg:np-decomposition}
%%   \begin{algorithmic}[1]
%%     \Require 向量流图中\texttt{map}原语指向的结点$V$
%%     \Ensure 由\texttt{map}原语生成$V$的操作中包含的任务组成的队列$TQ$
%%     \Function{np-decompsite}{$V, TQ$}
%%     \State $sop \leftarrow get\_scalar\_op(v)$
%%     \If{$contain\_vp(sop) = TRUE$}
%%     \For{$element$ in $get\_source\_vector(V)$}
%%     \State $TQ \leftarrow$ \Call{np-decompsite}{$make\_node(sop, element), TQ$}
%%     \EndFor
%%     \Else
%%     \State $TQ \leftarrow$ \Call{generate-task}{$V$}
%%     \EndIf
%%     \State \Return{$TQ$}
%%     \EndFunction
%%   \end{algorithmic}
%% \end{algorithm}

\begin{quotation}
  \kai{
    经过嵌套并行分解处理,n-body问题中包含嵌套并行操作的表达式\texttt{map (calcSingleAcc cell) cells}
    实际上被翻译成一组数量为\texttt{cells}长度的一维向量操作,如图\ref{fig:np-decomposition}。
    从图中可以看出,嵌套并行分解生成了一组新的流,这些流的结构相同,仅存在一些参数上的差异,
    这种参数差异反映了内层并行操作对外层向量数据不同位置元素的引用。
  }
\end{quotation}
\begin{figure}
  \centering
  \includegraphics[width=\linewidth]{np-decomposition}
  \caption{嵌套并行分解操作}
  \label{fig:np-decomposition}
\end{figure}

\section{小结}
本章详细说明了Rat程序的编译过程,以及Rat编译实现的关键技术。

中间语言Core的设计是整个编译实现技术中非常重要的环节。
Core语言指令集包括一组向量原语,
大量标量原语以及少量辅助指令。
标量原语为Rat提供了数据并行能力,标量原语提供了数值计算能力。
Core语言适合描述数据并行问题,同时它的标量原语可以在并行硬件上高效实现。

流探测技术是Rat编译实现中另一个关键技术,
它为程序的自动并行化提供了支持。向量流图与流驱动模型是一种
发现程序并行潜力的通用方法,与具体问题无关。
而流探测算法能够将问题转化成向量流图,从而使用流驱动模型
产生并行任务。

同时,流探测技术还是Rat编译期执行高层流优化措施与产生C语言代码的基础。

