\chapter{结束语}
本章先对全文工作进行总结,指出课题中存在的不足,并对将来的工作进行展望。

\section{全文工作总结}
论文设计了一种函数式并行语言Rat,并设计了Rat语言在CPU/GPU异构计算机系统中的编译实现方案。

Rat采用函数式语言设计,语法简洁,语义清晰,抽象层次高,表达能力强。
使用Rat编程,用户无需关心存储管理、线程管理、线程通信等硬件实现细节,
只需要关注问题本身的逻辑。

Rat面向数据并行问题,定位于科学工程计算,使用并行向量操作描述数据并行特性。
这种语言设计实质上是SPMD、STMD编程模型的一种高层抽象,适合在并行硬件上实现。

中间语言Core的设计与流探测技术的应用,将数据并行问题转化为数据流表示,最终
采用数据流驱动的方式实现任务自动并行化。同时,在提出了两种实施在流上的高层优化技术。

Rat的运行时系统简化了并行编程任务。
它负责管理GPU端存储器空间的分配与回收,并提出了若干提高存储空间利用率的方法。
它负责任务的动态调度,提出了以“单行流”为调度单位的创新设计,非常适合GPU的软硬件模型,
能够充分利用GPU可用计算资源,同时能够保证调度到GPU上的任务不存在数据依赖从而可以并行执行。

\section{不足与展望}
本课题存在的不足包括以下几个方面。

首先,嵌套并行分解技术可能会造成大量小规模任务,在这种情况下,
将任务发送到到GPU上造成的附加开销可能会严重影响整个程序性能。
这就需要采取一些措施,将多个小规模任务合并成大任务,
一次性发送到GPU执行,启动大量Kernel造成的附加开销。
这可以从NESL\upcite{Blelloch1995}以及Haskell的嵌套并行技术\upcite{Chakravarty2000}
获得某些启发,使用嵌套并行一维化代替分解。

第二,由于在CPU端调用GPU全局存储器的管理API,会导致所有GPU任务流上的计算任务
被中断,所以,运行时系统采用了启动时一次申请大块内存的方式避免在程序运行过程中
调用内存分配函数。现在,这个初始内存的大小只能通过用户指定,初始内存过大会造成
资源浪费,过小又会造成计算任务被中断损失性能。在下一步工作中,
需要考虑如何通过编译期分析,对整个程序运行过程中所需内存总量进行合理估计。

虽然论文中提出的实现方案针对CPU/GPU异构计算机系统,但原理上Rat在
共享内存多处理器计算机、共享内存多核处理器计算机、分布式内存计算机集群上都
能够有效实现,只需要在这些系统上分别实现有效的运行时系统即可。
